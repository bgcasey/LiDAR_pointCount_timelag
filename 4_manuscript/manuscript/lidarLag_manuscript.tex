\documentclass[manuscript, 3p, authoryear]{elsarticle} %review=doublespace preprint=single 5p=2 column
%%% Begin My package additions %%%%%%%%%%%%%%%%%%%

\usepackage[hyphens]{url}

  \journal{Remote Sensing in Ecology and Conservation} % Sets Journal name

\usepackage{lineno} % add
  \linenumbers % turns line numbering on

\usepackage{graphicx}
%%%%%%%%%%%%%%%% end my additions to header

\usepackage[T1]{fontenc}
\usepackage{lmodern}
\usepackage{amssymb,amsmath}
\usepackage{ifxetex,ifluatex}
\usepackage{fixltx2e} % provides \textsubscript
% use upquote if available, for straight quotes in verbatim environments
\IfFileExists{upquote.sty}{\usepackage{upquote}}{}
\ifnum 0\ifxetex 1\fi\ifluatex 1\fi=0 % if pdftex
  \usepackage[utf8]{inputenc}
\else % if luatex or xelatex
  \usepackage{fontspec}
  \ifxetex
    \usepackage{xltxtra,xunicode}
  \fi
  \defaultfontfeatures{Mapping=tex-text,Scale=MatchLowercase}
  \newcommand{\euro}{€}
\fi
% use microtype if available
\IfFileExists{microtype.sty}{\usepackage{microtype}}{}

\ifxetex
  \usepackage[setpagesize=false, % page size defined by xetex
              unicode=false, % unicode breaks when used with xetex
              xetex]{hyperref}
\else
  \usepackage[unicode=true]{hyperref}
\fi
\hypersetup{breaklinks=true,
            bookmarks=true,
            pdfauthor={},
            pdftitle={The influence of LiDAR acquisition time lag on bird species distribution models},
            colorlinks=true,
            urlcolor=blue,
            linkcolor=magenta,
            pdfborder={0 0 0}}

\setcounter{secnumdepth}{5}
% Pandoc toggle for numbering sections (defaults to be off)


% tightlist command for lists without linebreak
\providecommand{\tightlist}{%
  \setlength{\itemsep}{0pt}\setlength{\parskip}{0pt}}

% From pandoc table feature
\usepackage{longtable,booktabs,array}
\usepackage{calc} % for calculating minipage widths
% Correct order of tables after \paragraph or \subparagraph
\usepackage{etoolbox}
\makeatletter
\patchcmd\longtable{\par}{\if@noskipsec\mbox{}\fi\par}{}{}
\makeatother
% Allow footnotes in longtable head/foot
\IfFileExists{footnotehyper.sty}{\usepackage{footnotehyper}}{\usepackage{footnote}}
\makesavenoteenv{longtable}

% Pandoc citation processing
\newlength{\cslhangindent}
\setlength{\cslhangindent}{1.5em}
\newlength{\csllabelwidth}
\setlength{\csllabelwidth}{3em}
\newlength{\cslentryspacingunit} % times entry-spacing
\setlength{\cslentryspacingunit}{\parskip}
% for Pandoc 2.8 to 2.10.1
\newenvironment{cslreferences}%
  {}%
  {\par}
% For Pandoc 2.11+
\newenvironment{CSLReferences}[2] % #1 hanging-ident, #2 entry spacing
 {% don't indent paragraphs
  \setlength{\parindent}{0pt}
  % turn on hanging indent if param 1 is 1
  \ifodd #1
  \let\oldpar\par
  \def\par{\hangindent=\cslhangindent\oldpar}
  \fi
  % set entry spacing
  \setlength{\parskip}{#2\cslentryspacingunit}
 }%
 {}
\usepackage{calc}
\newcommand{\CSLBlock}[1]{#1\hfill\break}
\newcommand{\CSLLeftMargin}[1]{\parbox[t]{\csllabelwidth}{#1}}
\newcommand{\CSLRightInline}[1]{\parbox[t]{\linewidth - \csllabelwidth}{#1}\break}
\newcommand{\CSLIndent}[1]{\hspace{\cslhangindent}#1}




\begin{document}


\begin{frontmatter}

  \title{The influence of LiDAR acquisition time lag on bird species distribution models}
    \author[University of Alberta]{Brendan Casey*%
  \corref{cor1}%
  \fnref{1}}
   \ead{bgcasey@ualberta.ca} 
    \author[University of Alberta]{Erin Bayne}
   \ead{bayne@ualberta.ca} 
      \affiliation[University of Alberta]{Department of Biological Sciences, CW 405, Biological Sciences Bldg., Edmonton, AB, T6G 2E9}
    \cortext[cor1]{Corresponding author}
    \fntext[1]{This is the first author footnote.}
    \fntext[2]{Another author footnote.}
  
  \begin{abstract}
  This is the abstract.

  It consists of two paragraphs.
  \end{abstract}
    \begin{keyword}
    Avian \sep Boreal \sep Forestry \sep 
    LiDAR
  \end{keyword}
  
 \end{frontmatter}

\hypertarget{introduction}{%
\section{Introduction}\label{introduction}}

LiDAR has the potential to improve bird models by providing high resolution structural covariates which, when paired with bird monitoring data, can give insight into bird-habitat relationships {[}\protect\hyperlink{ref-Bradbury2005}{1}{]}. However, LiDAR acquisitions do not always coincide in time with point count surveys. It is unclear how much this temporal misalignment can influence bird distribution models that use LiDAR derived predictor variables. As disturbance-succession cycles change vegetation structure, eventually LiDAR metrics will no longer reflect ground conditions. Their usefulness as explanatory variables will degrade {[}\protect\hyperlink{ref-VierlingSwift2014}{2}{]}. Here, we evaluated how time lag between LiDAR acquisitions and bird surveys influenced model robustness for early successional, mature forest, and forest generalist birds.

The composition and structure of forests are changing in response to climate change, shifts to natural disturbance regimes, and increasing industrial development {[}\protect\hyperlink{ref-Brandt2013}{3}{]}. Predictive models linking field observations to environmental variables can reveal how birds respond to these changes {[}\protect\hyperlink{ref-Carrillo-Rubio2014}{4}--\protect\hyperlink{ref-He2015}{7}{]}. Broadly known as species distribution models (SDMs), this family of statistical methods predict bird distributions by comparing habitat where individuals were observed against habitat where they were absent {[}\protect\hyperlink{ref-Guisan2005}{8}{]}. SDMs and resulting predictive distribution maps are used to understand bird habitat preferences and the drivers of broad scale population declines and have applications in conservation management planning and environmental impact assessments {[}\protect\hyperlink{ref-englerAvianSDMsCurrent2017}{5},\protect\hyperlink{ref-franklinMappingSpeciesDistributions2010}{9}{]}.

Many factors influence the predictive capacity of SDMs, but the inclusion of ecologically relevant spatial covariates are key drivers of model accuracy {[}\protect\hyperlink{ref-Franklin1995}{10}--\protect\hyperlink{ref-fourcadePaintingsPredictDistribution2018}{12}{]}. Bird SDMs often rely on categorical predictors derived from digital maps delineating land cover, vegetation composition, and human footprint. While useful, they often miss key forest features driving habitat selection, namely those related to vegetation structure.

Vegetation structure influences the abundance, distribution, and behavior of birds {[}\protect\hyperlink{ref-Davies2014a}{14}{]}. The height and density of vegetation influence where birds perch, feed, and reproduce {[}\protect\hyperlink{ref-Bradbury2005}{1}{]} by mediating microclimates, providing shelter from weather {[}\protect\hyperlink{ref-CarrascalDiaz2006}{15}{]}, concealment from predators {[}\protect\hyperlink{ref-GotmarkBlomqvist1995}{16}{]}, and creating habitat for insect prey {[}\protect\hyperlink{ref-halajImportanceHabitatStructure2000}{17}{]}. Light Detection and Ranging (LiDAR) can characterize these three-dimensional forest structures {[}\protect\hyperlink{ref-limLiDARRemoteSensing2003}{18}{]}. Common LiDAR derived metrics correspond with vegetation height, cover, structural complexity, and density of forest strata {[}\protect\hyperlink{ref-Davies2014a}{14},\protect\hyperlink{ref-Lefsky2002}{19}--\protect\hyperlink{ref-Renner2018}{22}{]}. Used as predictor variables, LiDAR metrics can improve the predictive power of bird SDMS {[}\protect\hyperlink{ref-Bae2014}{26}{]}.

Publicly funded regional LIDAR data and space-based sensors like NASA's Ice, Cloud and Land Elevation Satellite-2 (ICESat-2) and Global Ecosystem Dynamics Investigation (GEDI), have made large amounts of wall-to-wall structural data available to researchers {[}\protect\hyperlink{ref-coopsForestStructureHabitat2016}{27}--\protect\hyperlink{ref-abdalatiICESat2LaserAltimetry2010}{29}{]}. However, LiDAR continues to be under-used in bird ecology. The limited temporal resolution of most LiDAR products may be a factor. LiDAR is often limited to a single season, with long multiyear gaps between repeat surveys. Temporal misalignment between wildlife surveys and LiDAR is common.

Temporal misalignment occurs when wildlife surveys and LiDAR acquisitions are done at different times {[}\protect\hyperlink{ref-VierlingSwift2014}{2}{]}. It's unclear how much temporal misalignment influences the performance of LiDAR based SDMs. Disturbance-succession cycles drive changes in vegetation structure, and eventually, LiDAR gathered over a season will no longer reflect ground conditions. This can occur when the surveyed forest transitions between stages of stand development, e.g.~from stand initiation to stem exclusion {[}\protect\hyperlink{ref-babcockModelingForestBiomass2016}{30},\protect\hyperlink{ref-brassardStandStructureComposition2010}{31}{]}. Temporal misalignment can impact the power of bird SDMs as successional changes in forest structure influence habitat selection by birds {[}\protect\hyperlink{ref-sittersAssociationsOccupancyHabitat2014}{32}{]}.

Consider Canada's boreal forests. It is a dynamic successional mosaic driven by forestry, fire, and energy exploration {[}\protect\hyperlink{ref-Brandt2013}{3},\protect\hyperlink{ref-gauthierBorealForestHealth2015}{33}{]}. The landscape is a patchwork of early to late successional stands with distinct structural characteristics {[}\protect\hyperlink{ref-Bergeron2012}{34}{]} and bird communities {[}\protect\hyperlink{ref-Schieck2006}{35}{]}. In early successional forests, bird communities are dominated by species that nest and forage in open vegetation, wetlands, and shrubs, along with some habitat generalists. As trees regenerate and the stand's structural properties change, open habitat species give way to species associated with corresponding forest age classes and strata {[}\protect\hyperlink{ref-lestonLongtermChangesBoreal2018}{36}{]}.

Thus, succession occurring between LiDAR and wildlife surveys may influence SDM performance. Consequently, LiDAR's usefulness as a source of explanatory variables can degrade as temporal misalignment increases. For researchers pairing LiDAR covariates with long-term wildlife survey data, this can lead to a trade-off: (1) minimize temporal misalignment by reducing the sample size to survey data gathered near the time of the LiDAR acquisition, or (2) maximize sample size and risk sacrificing model power.

To inform this trade-off, we addressed the question of how much temporal misalignment is acceptable in LiDAR based SDMs. Our objectives were to (1) evaluate how the time lag between LiDAR acquisitions and bird surveys influence the performance of SDMs across a gradient of 0 to 15 years, (2) compare the influence of temporal misalignment on models for early successional, mid-successional, mature forest, and forest generalist birds, and (3) assess how differences in resultant predictive distribution maps correlate with forest age.

The effects of temporal misalignment on SDMs will likely vary by habitat type (e.g.~forest age, disturbance history, and dominant vegetation) and the life history characteristics of the study species. We predicted that the performance of SDMs will decrease with increased temporal misalignment and that the magnitude of change will vary according to the habitat associations of the focal species. We predicted that (1) SDMs for early successional specialists, Mourning Warbler (\emph{Geothlypis philadelphia}) and White-throated Sparrow (\emph{Zonotrichia albicollis}), would be most affected by temporal misalignment because of faster vertical growth rates of establishment trees and loss of dense shrub layers {[}\protect\hyperlink{ref-mccarthy2001gap}{37}--\protect\hyperlink{ref-pitocchelliMourningWarblerGeothlypis2020}{39}{]}. (2) SDMs for mid-seral species like American Redstart (\emph{Setophaga ruticilla}) that are associated with dense midstory vegetation, would see moderate declines in performance as temporal misalignment increases due to self-thinning during the stem exclusion stage of succession {[}\protect\hyperlink{ref-brassardStandStructureComposition2010}{31},\protect\hyperlink{ref-sherryAmericanRedstartSetophaga2020a}{40}{]}. And (3) mature forest associates, Black-throated Green Warbler (\emph{Setophaga virens}), will be least effected by temporal misalignment as the processes effecting mature forest canopy structure (insect defoliation and windthrow) happen at too small a scale to effect overall model performance {[}\protect\hyperlink{ref-VierlingSwift2014}{2},\protect\hyperlink{ref-morseBlackthroatedGreenWarbler2020}{41}{]}. For all species, we predicted that differences in distribution maps will be negatively correlated with forest age.

\hypertarget{references}{%
\section*{References}\label{references}}
\addcontentsline{toc}{section}{References}

\hypertarget{refs}{}
\begin{CSLReferences}{0}{0}
\leavevmode\vadjust pre{\hypertarget{ref-Bradbury2005}{}}%
\CSLLeftMargin{{[}1{]} }%
\CSLRightInline{R.B. Bradbury, R.A. Hill, D.C. Mason, S.A. Hinsley, J.D. Wilson, H. Balzter, G.Q.A. Anderson, M.J. Whittingham, I.J. Davenport, P.E. Bellamy, Modelling relationships between birds and vegetation structure using airborne {LiDAR} data: A review with case studies from agricultural and woodland environments, Ibis. 147 (2005) 443--452. https://doi.org/\href{https://doi.org/10.1111/j.1474-919x.2005.00438.x}{10.1111/j.1474-919x.2005.00438.x}.}

\leavevmode\vadjust pre{\hypertarget{ref-VierlingSwift2014}{}}%
\CSLLeftMargin{{[}2{]} }%
\CSLRightInline{K.T. Vierling, C.E. Swift, A.T. Hudak, J.C. Vogeler, L.A. Vierling, How much does the time lag between wildlife field-data collection and {LiDAR-data} acquisition matter for studies of animal distributions? {A} case study using bird communities, Remote Sensing Letters. 5 (2014) 185--193. https://doi.org/\href{https://doi.org/10.1080/2150704X.2014.891773}{10.1080/2150704X.2014.891773}.}

\leavevmode\vadjust pre{\hypertarget{ref-Brandt2013}{}}%
\CSLLeftMargin{{[}3{]} }%
\CSLRightInline{J.P. Brandt, M.D. Flannigan, D.G. Maynard, I.D. Thompson, W.J.A. Volney, An introduction to {Canada}'s boreal zone: {Ecosystem} processes, health, sustainability, and environmental issues, Environmental Reviews. 21 (2013) 207--226. https://doi.org/\href{https://doi.org/10.1139/er-2013-0040}{10.1139/er-2013-0040}.}

\leavevmode\vadjust pre{\hypertarget{ref-Carrillo-Rubio2014}{}}%
\CSLLeftMargin{{[}4{]} }%
\CSLRightInline{E. Carrillo-Rubio, M. Kéry, S.J. Morreale, P.J. Sullivan, B. Gardner, E.G. Cooch, J.P. Lassoie, Use of multispecies occupancy models to evaluate the response of bird communities to forest degradation associated with logging, Conservation Biology. 28 (2014) 1034--1044. https://doi.org/\href{https://doi.org/10.1111/cobi.12261}{10.1111/cobi.12261}.}

\leavevmode\vadjust pre{\hypertarget{ref-englerAvianSDMsCurrent2017}{}}%
\CSLLeftMargin{{[}5{]} }%
\CSLRightInline{J.O. Engler, D. Stiels, K. Schidelko, D. Strubbe, P. Quillfeldt, M. Brambilla, Avian {SDMs}: Current state, challenges, and opportunities, Journal of Avian Biology. 48 (2017) 1483--1504. https://doi.org/\href{https://doi.org/10.1111/jav.01248}{10.1111/jav.01248}.}

\leavevmode\vadjust pre{\hypertarget{ref-guisanPredictiveHabitatDistribution2000}{}}%
\CSLLeftMargin{{[}6{]} }%
\CSLRightInline{A. Guisan, N.E. Zimmermann, Predictive habitat distribution models in ecology, Ecological Modelling. 135 (2000) 147--186. https://doi.org/\href{https://doi.org/10.1016/S0304-3800(00)00354-9}{10.1016/S0304-3800(00)00354-9}.}

\leavevmode\vadjust pre{\hypertarget{ref-He2015}{}}%
\CSLLeftMargin{{[}7{]} }%
\CSLRightInline{K.S. He, B.A. Bradley, A.F. Cord, D. Rocchini, M.-N.N. Tuanmu, S. Schmidtlein, W. Turner, M. Wegmann, N. Pettorelli, Will remote sensing shape the next generation of species distribution models?, Remote Sensing in Ecology and Conservation. 1 (2015) 4--18. https://doi.org/\href{https://doi.org/10.1002/rse2.7}{10.1002/rse2.7}.}

\leavevmode\vadjust pre{\hypertarget{ref-Guisan2005}{}}%
\CSLLeftMargin{{[}8{]} }%
\CSLRightInline{A. Guisan, W. Thuiller, Predicting species distribution: Offering more than simple habitat models, Ecology Letters. 8 (2005) 993--1009. https://doi.org/\href{https://doi.org/10.1111/j.1461-0248.2005.00792.x}{10.1111/j.1461-0248.2005.00792.x}.}

\leavevmode\vadjust pre{\hypertarget{ref-franklinMappingSpeciesDistributions2010}{}}%
\CSLLeftMargin{{[}9{]} }%
\CSLRightInline{J. Franklin, Mapping species distributions: Spatial inference and prediction, {Cambridge University Press}, 2010.}

\leavevmode\vadjust pre{\hypertarget{ref-Franklin1995}{}}%
\CSLLeftMargin{{[}10{]} }%
\CSLRightInline{J. Franklin, Predictive vegetation mapping: Geographic modelling of biospatial patterns in relation to environmental gradients, Progress in Physical Geography: Earth and Environment. 19 (1995) 474--499. https://doi.org/\href{https://doi.org/10.1177/030913339501900403}{10.1177/030913339501900403}.}

\leavevmode\vadjust pre{\hypertarget{ref-Vaughn2003}{}}%
\CSLLeftMargin{{[}11{]} }%
\CSLRightInline{I.P. Vaughn, S.J. Ormerod, Improving the quality of distribution models for conservation by addressing shortcomings in the field collection of training data, Conservation Biology. 17 (2003) 1601--1611. https://doi.org/\href{https://doi.org/10.1111/j.1523-1739.2003.00359.x}{10.1111/j.1523-1739.2003.00359.x}.}

\leavevmode\vadjust pre{\hypertarget{ref-fourcadePaintingsPredictDistribution2018}{}}%
\CSLLeftMargin{{[}12{]} }%
\CSLRightInline{Y. Fourcade, A.G. Besnard, J. Secondi, Paintings predict the distribution of species, or the challenge of selecting environmental predictors and evaluation statistics, Global Ecology and Biogeography. 27 (2018) 245--256. https://doi.org/\href{https://doi.org/10.1111/geb.12684}{10.1111/geb.12684}.}

\leavevmode\vadjust pre{\hypertarget{ref-MacArthurMacArthur1961}{}}%
\CSLLeftMargin{{[}13{]} }%
\CSLRightInline{R.H. MacArthur, J.W. MacArthur, On bird species diversity, Ecology. 42 (1961) 594--598. https://doi.org/\href{https://doi.org/10.2307/1932254}{10.2307/1932254}.}

\leavevmode\vadjust pre{\hypertarget{ref-Davies2014a}{}}%
\CSLLeftMargin{{[}14{]} }%
\CSLRightInline{A.B. Davies, G.P. Asner, Advances in animal ecology from {3D-LiDAR} ecosystem mapping, Trends in Ecology and Evolution. 29 (2014) 681--691. https://doi.org/\href{https://doi.org/10.1016/j.tree.2014.10.005}{10.1016/j.tree.2014.10.005}.}

\leavevmode\vadjust pre{\hypertarget{ref-CarrascalDiaz2006}{}}%
\CSLLeftMargin{{[}15{]} }%
\CSLRightInline{L. Carrascal, L. Diaz, Winter bird distribution in abiotic and habitat structural gradients: A case study with mediterranean montane oakwoods, Ecoscience. 13 (2006) 100--110.}

\leavevmode\vadjust pre{\hypertarget{ref-GotmarkBlomqvist1995}{}}%
\CSLLeftMargin{{[}16{]} }%
\CSLRightInline{F. Gotmark, D. Blomqvist, O.C. Johansson, J. Bergkvist, Nest {Site Selection}: {A Trade-Off} between {Concealment} and {View} of the {Surroundings}?, Journal of Avian Biology. 26 (1995) 305. https://doi.org/\href{https://doi.org/10.2307/3677045}{10.2307/3677045}.}

\leavevmode\vadjust pre{\hypertarget{ref-halajImportanceHabitatStructure2000}{}}%
\CSLLeftMargin{{[}17{]} }%
\CSLRightInline{J. Halaj, D. Ross, A. Moldenke, Importance of habitat structure to the arthropod food-web in {Douglas-fir} canopies, Oikos (Copenhagen, Denmark). 90 (2000) 139--152. https://doi.org/\href{https://doi.org/10.1034/j.1600-0706.2000.900114.x}{10.1034/j.1600-0706.2000.900114.x}.}

\leavevmode\vadjust pre{\hypertarget{ref-limLiDARRemoteSensing2003}{}}%
\CSLLeftMargin{{[}18{]} }%
\CSLRightInline{K. Lim, P. Treitz, M. Wulder, B. St-Onge, M. Flood, {LiDAR} remote sensing of forest structure, Progress in Physical Geography. 27 (2003) 88--106. https://doi.org/\href{https://doi.org/10.1191/0309133303pp360ra}{10.1191/0309133303pp360ra}.}

\leavevmode\vadjust pre{\hypertarget{ref-Lefsky2002}{}}%
\CSLLeftMargin{{[}19{]} }%
\CSLRightInline{M.A. Lefsky, W.B. Cohen, G.G. Parker, D.J. Harding, G.G. Parker, D.J. Harding, Lidar remote sensing for ecosystem studies, BioScience. 52 (2002) 19--30. https://doi.org/\href{https://doi.org/10.1641/0006-3568(2002)052\%5B0019:lrsfes\%5D2.0.co;2}{10.1641/0006-3568(2002)052{[}0019:lrsfes{]}2.0.co;2}.}

\leavevmode\vadjust pre{\hypertarget{ref-Bae2018}{}}%
\CSLLeftMargin{{[}20{]} }%
\CSLRightInline{S. Bae, J. Müller, D. Lee, K.T. Vierling, J.C. Vogeler, L.A. Vierling, A.T. Hudak, H. Latifi, S. Thorn, Taxonomic, functional, and phylogenetic diversity of bird assemblages are oppositely associated to productivity and heterogeneity in temperate forests, Remote Sensing of Environment. 215 (2018) 145--156. https://doi.org/\href{https://doi.org/10.1016/j.rse.2018.05.031}{10.1016/j.rse.2018.05.031}.}

\leavevmode\vadjust pre{\hypertarget{ref-Kortmann2018}{}}%
\CSLLeftMargin{{[}21{]} }%
\CSLRightInline{M. Kortmann, M. Heurich, H. Latifi, S. Roesner, R. Seidl, J. Mueller, S. Thorn, S. Rösner, R. Seidl, J. Müller, S. Thorn, Forest structure following natural disturbances and early succession provides habitat for two avian flagship species, capercaillie ({Tetrao} urogallus) and hazel grouse ({Tetrastes} bonasia), Biological Conservation. 226 (2018) 81--91. https://doi.org/\href{https://doi.org/10.1016/j.biocon.2018.07.014}{10.1016/j.biocon.2018.07.014}.}

\leavevmode\vadjust pre{\hypertarget{ref-Renner2018}{}}%
\CSLLeftMargin{{[}22{]} }%
\CSLRightInline{S.C. Renner, M. Suarez-Rubio, S. Kaiser, J. Nieschulze, E.K.V.V. Kalko, M. Tschapka, K. Jung, Divergent response to forest structure of two mobile vertebrate groups, Forest Ecology and Management. 415--416 (2018) 129--138. https://doi.org/\href{https://doi.org/10.1016/j.foreco.2018.02.028}{10.1016/j.foreco.2018.02.028}.}

\leavevmode\vadjust pre{\hypertarget{ref-farrellUsingLiDARderivedVegetation2013b}{}}%
\CSLLeftMargin{{[}23{]} }%
\CSLRightInline{S.L. Farrell, B.A. Collier, K.L. Skow, A.M. Long, A.J. Campomizzi, M.L. Morrison, K.B. Hays, R.N. Wilkins, Using {LiDAR-derived} vegetation metrics for high-resolution, species distribution models for conservation planning, Ecosphere. 4 (2013) 42. https://doi.org/\href{https://doi.org/10.1890/ES12-000352.1}{10.1890/ES12-000352.1}.}

\leavevmode\vadjust pre{\hypertarget{ref-ficetolaHowManyPredictors2014}{}}%
\CSLLeftMargin{{[}24{]} }%
\CSLRightInline{G.F. Ficetola, A. Bonardi, C.A. Mücher, N.L.M. Gilissen, E. Padoa-Schioppa, How many predictors in species distribution models at the landscape scale? {Land} use versus {LiDAR-derived} canopy height, International Journal of Geographical Information Science. 28 (2014) 1723--1739. https://doi.org/\href{https://doi.org/10.1080/13658816.2014.891222}{10.1080/13658816.2014.891222}.}

\leavevmode\vadjust pre{\hypertarget{ref-clawgesUseAirborneLidar2008}{}}%
\CSLLeftMargin{{[}25{]} }%
\CSLRightInline{R. Clawges, K. Vierling, L. Vierling, E. Rowell, The use of airborne lidar to assess avian species diversity, density, and occurrence in a pine/aspen forest, Remote Sensing of Environment. 112 (2008) 2064--2073. https://doi.org/\href{https://doi.org/10.1016/j.rse.2007.08.023}{10.1016/j.rse.2007.08.023}.}

\leavevmode\vadjust pre{\hypertarget{ref-Bae2014}{}}%
\CSLLeftMargin{{[}26{]} }%
\CSLRightInline{S. Bae, B. Reineking, M. Ewald, J. Mueller, Comparison of airborne lidar, aerial photography, and field surveys to model the habitat suitability of a cryptic forest species \textendash{} the hazel grouse, International Journal of Remote Sensing. 35 (2014) 6469--6489. https://doi.org/\href{https://doi.org/10.1080/01431161.2014.955145}{10.1080/01431161.2014.955145}.}

\leavevmode\vadjust pre{\hypertarget{ref-coopsForestStructureHabitat2016}{}}%
\CSLLeftMargin{{[}27{]} }%
\CSLRightInline{N.C. Coops, P. Tompaski, W. Nijland, G.J.M. Rickbeil, S.E. Nielsen, C.W. Bater, J.J. Stadt, A forest structure habitat index based on airborne laser scanning data, Ecological Indicators. 67 (2016) 346--357. https://doi.org/\href{https://doi.org/10.1016/j.ecolind.2016.02.057}{10.1016/j.ecolind.2016.02.057}.}

\leavevmode\vadjust pre{\hypertarget{ref-dubayahGlobalEcosystemDynamics2020a}{}}%
\CSLLeftMargin{{[}28{]} }%
\CSLRightInline{R. Dubayah, J.B. Blair, S. Goetz, L. Fatoyinbo, M. Hansen, S. Healey, M. Hofton, G. Hurtt, J. Kellner, S. Luthcke, J. Armston, H. Tang, L. Duncanson, S. Hancock, P. Jantz, S. Marselis, P.L. Patterson, W. Qi, C. Silva, The {Global Ecosystem Dynamics Investigation}: {High-resolution} laser ranging of the {Earth}'s forests and topography, Science of Remote Sensing. 1 (2020) 100002. https://doi.org/\href{https://doi.org/10.1016/j.srs.2020.100002}{10.1016/j.srs.2020.100002}.}

\leavevmode\vadjust pre{\hypertarget{ref-abdalatiICESat2LaserAltimetry2010}{}}%
\CSLLeftMargin{{[}29{]} }%
\CSLRightInline{W. Abdalati, H.J. Zwally, R. Bindschadler, B. Csatho, S.L. Farrell, H.A. Fricker, D. Harding, R. Kwok, M. Lefsky, T. Markus, A. Marshak, T. Neumann, S. Palm, B. Schutz, B. Smith, J. Spinhirne, C. Webb, The {ICESat-2 Laser Altimetry Mission}, Proceedings of the IEEE. 98 (2010) 735--751. https://doi.org/\href{https://doi.org/10.1109/JPROC.2009.2034765}{10.1109/JPROC.2009.2034765}.}

\leavevmode\vadjust pre{\hypertarget{ref-babcockModelingForestBiomass2016}{}}%
\CSLLeftMargin{{[}30{]} }%
\CSLRightInline{C. Babcock, A.O. Finley, B.D. Cook, A. Weiskittel, C.W. Woodall, Modeling forest biomass and growth: {Coupling} long-term inventory and {LiDAR} data, Remote Sensing of Environment. 182 (2016) 1--12. https://doi.org/\href{https://doi.org/10.1016/j.rse.2016.04.014}{10.1016/j.rse.2016.04.014}.}

\leavevmode\vadjust pre{\hypertarget{ref-brassardStandStructureComposition2010}{}}%
\CSLLeftMargin{{[}31{]} }%
\CSLRightInline{B.W. Brassard, H.Y.H. Chen, Stand {Structure} and {Composition Dynamics} of {Boreal Mixedwood Forest}: {Implications} for {Forest Management}, 2010.}

\leavevmode\vadjust pre{\hypertarget{ref-sittersAssociationsOccupancyHabitat2014}{}}%
\CSLLeftMargin{{[}32{]} }%
\CSLRightInline{H. Sitters, F. Christie, J. Di Stefano, M. Swan, P. Collins, A. York, Associations between occupancy and habitat structure can predict avian responses to disturbance: {Implications} for conservation management, FOREST ECOLOGY AND MANAGEMENT. 331 (2014) 227--236. https://doi.org/\href{https://doi.org/10.1016/j.foreco.2014.08.013}{10.1016/j.foreco.2014.08.013}.}

\leavevmode\vadjust pre{\hypertarget{ref-gauthierBorealForestHealth2015}{}}%
\CSLLeftMargin{{[}33{]} }%
\CSLRightInline{S. Gauthier, P. Bernier, T. Kuuluvainen, A.Z. Shvidenko, D.G. Schepaschenko, Boreal forest health and global change, Science. 349 (2015) 819--822. https://doi.org/\href{https://doi.org/10.1126/science.aaa9092}{10.1126/science.aaa9092}.}

\leavevmode\vadjust pre{\hypertarget{ref-Bergeron2012}{}}%
\CSLLeftMargin{{[}34{]} }%
\CSLRightInline{Y. Bergeron, N.J. Fenton, Boreal forests of eastern {Canada} revisited: {Old} growth, nonfire disturbances, forest succession, and biodiversity, Botany. 90 (2012) 509--523. https://doi.org/\href{https://doi.org/10.1139/B2012-034}{10.1139/B2012-034}.}

\leavevmode\vadjust pre{\hypertarget{ref-Schieck2006}{}}%
\CSLLeftMargin{{[}35{]} }%
\CSLRightInline{J. Schieck, S.J. Song, Changes in bird communities throughout succession following fire and harvest in boreal forests of western {North America}: {Literature} review and meta-analyses, Canadian Journal of Forest Research. 36 (2006) 1299--1318. https://doi.org/\href{https://doi.org/10.1139/X06-017}{10.1139/X06-017}.}

\leavevmode\vadjust pre{\hypertarget{ref-lestonLongtermChangesBoreal2018}{}}%
\CSLLeftMargin{{[}36{]} }%
\CSLRightInline{L. Leston, E. Bayne, F. Schmiegelow, Long-term changes in boreal forest occupancy within regenerating harvest units, Forest Ecology and Management. 421 (2018) 40--53. https://doi.org/\href{https://doi.org/10.1016/J.FORECO.2018.02.029}{10.1016/J.FORECO.2018.02.029}.}

\leavevmode\vadjust pre{\hypertarget{ref-mccarthy2001gap}{}}%
\CSLLeftMargin{{[}37{]} }%
\CSLRightInline{J. McCarthy, Gap dynamics of forest trees: A review with particular attention to boreal forests, Environmental Reviews. 9 (2001) 1--59.}

\leavevmode\vadjust pre{\hypertarget{ref-fallsWhitethroatedSparrowZonotrichia2020}{}}%
\CSLLeftMargin{{[}38{]} }%
\CSLRightInline{J. Falls, J. Kopachena, White-throated {Sparrow} ({Zonotrichia} albicollis), version 1.0, in: Birds of the {World} ({A}. {F}. {Poole}, {Editor}), {Cornell Lab of Ornithology}, {Ithaca, NY, USA}, 2020. https://doi.org/\href{https://doi.org/10.2173/bow.whtspa.01}{10.2173/bow.whtspa.01}.}

\leavevmode\vadjust pre{\hypertarget{ref-pitocchelliMourningWarblerGeothlypis2020}{}}%
\CSLLeftMargin{{[}39{]} }%
\CSLRightInline{J. Pitocchelli, Mourning {Warbler} ({Geothlypis} philadelphia), version 1.0, in: Birds of the {World} ({P}. {G}. {Rodewald}, {Editor}), {Cornell Lab of Ornithology}, {Ithaca, NY, USA}, 2020. https://doi.org/\href{https://doi.org/10.2173/bow.mouwar.01}{10.2173/bow.mouwar.01}.}

\leavevmode\vadjust pre{\hypertarget{ref-sherryAmericanRedstartSetophaga2020a}{}}%
\CSLLeftMargin{{[}40{]} }%
\CSLRightInline{T. Sherry, R. Holmes, P. Pyle, M. Patten, P. Rodewald, American redstart (setophaga ruticilla), version 1.0, in: Birds of the {World} ({P}. {G}. {Rodewald}, {Editor}), {Cornell Lab of Ornithology}, {Ithaca, NY, USA}, 2020. https://doi.org/\href{https://doi.org/10.2173/bow.amered.01}{10.2173/bow.amered.01}.}

\leavevmode\vadjust pre{\hypertarget{ref-morseBlackthroatedGreenWarbler2020}{}}%
\CSLLeftMargin{{[}41{]} }%
\CSLRightInline{D. Morse, A. Poole, Black-throated {Green Warbler} ({Setophaga} virens), version 1.0, in: Birds of the {World} ({P}. {G}. {Rodewald}, {Editor}), {Cornell Lab of Ornithology}, {Ithaca, NY, USA}, 2020. https://doi.org/\href{https://doi.org/10.2173/btnwar.01}{10.2173/btnwar.01}.}

\end{CSLReferences}

\pagebreak


\end{document}
