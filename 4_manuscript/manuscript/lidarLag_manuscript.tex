\documentclass[preprint, 3p,
authoryear]{elsarticle} %review=doublespace preprint=single 5p=2 column
%%% Begin My package additions %%%%%%%%%%%%%%%%%%%

\usepackage[hyphens]{url}

  \journal{An awesome journal} % Sets Journal name

\usepackage{lineno} % add

\usepackage{graphicx}
%%%%%%%%%%%%%%%% end my additions to header

\usepackage[T1]{fontenc}
\usepackage{lmodern}
\usepackage{amssymb,amsmath}
\usepackage{ifxetex,ifluatex}
\usepackage{fixltx2e} % provides \textsubscript
% use upquote if available, for straight quotes in verbatim environments
\IfFileExists{upquote.sty}{\usepackage{upquote}}{}
\ifnum 0\ifxetex 1\fi\ifluatex 1\fi=0 % if pdftex
  \usepackage[utf8]{inputenc}
\else % if luatex or xelatex
  \usepackage{fontspec}
  \ifxetex
    \usepackage{xltxtra,xunicode}
  \fi
  \defaultfontfeatures{Mapping=tex-text,Scale=MatchLowercase}
  \newcommand{\euro}{€}
\fi
% use microtype if available
\IfFileExists{microtype.sty}{\usepackage{microtype}}{}
\usepackage[]{natbib}
\bibliographystyle{plainnat}

\ifxetex
  \usepackage[setpagesize=false, % page size defined by xetex
              unicode=false, % unicode breaks when used with xetex
              xetex]{hyperref}
\else
  \usepackage[unicode=true]{hyperref}
\fi
\hypersetup{breaklinks=true,
            bookmarks=true,
            pdfauthor={},
            pdftitle={Short Paper},
            colorlinks=false,
            urlcolor=blue,
            linkcolor=magenta,
            pdfborder={0 0 0}}

\setcounter{secnumdepth}{5}
% Pandoc toggle for numbering sections (defaults to be off)


% tightlist command for lists without linebreak
\providecommand{\tightlist}{%
  \setlength{\itemsep}{0pt}\setlength{\parskip}{0pt}}

% From pandoc table feature
\usepackage{longtable,booktabs,array}
\usepackage{calc} % for calculating minipage widths
% Correct order of tables after \paragraph or \subparagraph
\usepackage{etoolbox}
\makeatletter
\patchcmd\longtable{\par}{\if@noskipsec\mbox{}\fi\par}{}{}
\makeatother
% Allow footnotes in longtable head/foot
\IfFileExists{footnotehyper.sty}{\usepackage{footnotehyper}}{\usepackage{footnote}}
\makesavenoteenv{longtable}





\begin{document}


\begin{frontmatter}

  \title{Short Paper}
    \author[Some Institute of Technology]{Alice Anonymous%
  \corref{cor1}%
  \fnref{1}}
   \ead{alice@example.com} 
    \author[Another University]{Bob Security}
   \ead{bob@example.com} 
    \author[Another University]{Cat Memes%
  %
  \fnref{2}}
   \ead{cat@example.com} 
    \author[Some Institute of Technology]{Derek Zoolander%
  %
  \fnref{2}}
   \ead{derek@example.com} 
      \affiliation[Some Institute of Technology]{Department, Street,
City, State, Zip}
    \affiliation[Another University]{Department, Street, City, State,
Zip}
    \cortext[cor1]{Corresponding author}
    \fntext[1]{This is the first author footnote.}
    \fntext[2]{Another author footnote.}
  
  \begin{abstract}
  This is the abstract.

  It consists of two paragraphs.
  \end{abstract}
    \begin{keyword}
    keyword1 \sep 
    keyword2
  \end{keyword}
  
 \end{frontmatter}

LiDAR has the potential to improve bird models by providing high
resolution structural covariates which, when paired with bird monitoring
data, can provide insight into bird-habitat relationships {[}REF{]}.
However, LiDAR acquisitions do not always coincide temporally with bird
surveys, and it is unclear how much these time lags influence models.
Disturbance-succession cycles change vegetation structure. Eventually,
LiDAR metrics will no longer reflect ground conditions, and their
usefulness as explanatory variables will degrade. Here, we evaluated how
the time lag between LiDAR acquisitions and bird surveys influenced
model robustness for early successional, mature forest, and generalist
birds.

The composition and structure of forests are changing in response to
climate change, shifts to natural disturbance regimes, and increasing
industrial development \citep{Brandt2013}. Predictive models linking
distribution, abundance, and community structure to select environmental
variables have been used to understand how forest birds respond to these
changes
\citep{Carrillo-Rubio2014, Engler2017a, guisanPredictiveHabitatDistribution2000, He2015}.
Broadly known as species distribution models(SDMs), this family of
statistical methods relate field observations (e.g.~occupancy or
abundance) with spatial covariates \citep{Guisan2005}. Find correlations
between detection data and eviromnetal covariates to predict\ldots{}
While methods vary, SDMs predict species occurrence by comparing the
habitat where individuals have been observed (via traditional human
point counts or autonomous bioacoustic monitoring) against habitat where
species are absent {[}Guisan2005{]}. With innovations in modelling
methods, computational power, and environmental monitoring, SDMs are an
important tool in ecology, conservation biology, and wildlife management
\citep{Elith2009}. SDMs and resulting predictive distribution maps are
often applied to bird research \citet{englerAvianSDMsCurrent2017}, and
have informed conservation management planning, environmental impact
assessments{[}REF{]}, and bird diversity modelling
{[}REF{]}\citep{englerAvianSDMsCurrent2017, franklinMappingSpeciesDistributions2010}.

\begin{center}\rule{0.5\linewidth}{0.5pt}\end{center}

See reviews by Engler et al. -\citet{englerAvianSDMsCurrent2017} and
Elith et al. -\citet{Elith2009}. Engler et al.
-\citet{englerAvianSDMsCurrent2017} presented a detailed overview of the
many applications of SDMs in the study of birds.

They provide insite into: 1. habitat characteristics necessary for
birds, 2. Predictions of distributions and forecast distributions under
changing scenarios. 3. Used to estimate species abundance and density.
Explore the patterns and processes driving distributions of species in
space.

Species distribution models can be used to link occurrence data with
environmental covariates to predict species distributions.

While SDMs encompass a variety of statistical approaches including
Generalized linear models, generalized additive models, and Bayesian
approaches. MOST often encorpprate presence-absence data from species
detection data from traditional human point counts or autonomous
bioacoustic monitoring. Most follow the same

Many factors influence the predictive capacity of SDMs, but the
inclusion of ecologically relevant spatial covariates are key drivers of
model accuracy
\citep{Franklin1995, Vaughn2003, fourcadePaintingsPredictDistribution2018}.
Bird SDMs often rely on categorical predictors derived from digital maps
delineating land cover, vegetation composition, and human footprint.
While these models are often supplemented with continuous bioclimatic
variables, they often miss important features driving bird response
{[}REF{]}. Continuous predictors from remote sensing can better describe
the mechanisms driving habitat selection
\citep{heWillRemoteSensing2015}. For example, spectral metrics linked to
vegetation are often representative of the shelter and food resources
used by birds. The normalized vegetation index (NDVI) derived from
Sentinel and Landsat satellites has been used to measure vegetation
productivity, habitat variability, and plant phenology
\citep{pettorelliNormalizedDifferenceVegetation2011}. And the normalized
burn ratio (NBR) can quantify disturbance severity and successional
recovery rates \citep{hislopUsingLandsatSpectral2018}. MODIS land
surface temperature (LST) has been used to predict bird response to
heatwaves \citep{albrightHeatWavesMeasured2011}.

\begin{longtable}[]{@{}
  >{\raggedright\arraybackslash}p{(\columnwidth - 0\tabcolsep) * \real{0.9306}}@{}}
\toprule()
\begin{minipage}[b]{\linewidth}\raggedright
maps and digital forest resource inventories (FRIs) supplement
bioclimatic variables; les. While these products may contain useful
metrics related to plant species composition and human footprint, and
canopy height, they often lack data on three-dimensional often lack more
detailed measures of important habitat features driving bird habitat
selection. They don't fully describe the mechanisms driving habitat
selection. Lack data Predictors from space-based optical remote sensors
are used most often and can provide more detailed levels of eco improve
on the class based habitat predictors of FRI
\citep{heWillRemoteSensing2015}. and extreme weather events
{[}Space-based land surface temperature and precipitation measurements
from MODIS have been used over terrestrial weather station data for
\_models {[}REF{]}. has been used to extract predictors related to
harvest and burn severity.while spectral indices derived from satellite
and aerial imagery can map land cover change and measure rates of
post-disturbance habitat recovery \citep{Northrup2019, Rittenhouse2010}
\end{minipage} \\
\midrule()
\endhead
In the boreal, fire, forestry harvests, and linear features like roads
and seismic lines effect forest composition, structure, and age. froim
one driven by fire to one dominated by forestry and energy exploration.
composition and structure of boreal forests are changing in response to
climate change, shifts to Natural disturbance regimes are being replaced
by forestry and industrial development \\
Timelag between the Uderstanding the temporal limitations of LiDAR is
important when selecting sources of environmental covariates. \\
Here, we evaluated how the time lag between LiDAR acquisitions and bird
surveys influence the performance of SDMs across a gradient of 0 to 15
years. We evaluated if the influence of time-lag varies between early
successional, mature forest, and forest generalist birds. Finally, we
assessed how differences in the resultant predictive distribution maps
correspond to forest age. \\
We hypothesized that the performance of models will decrease with
increased LiDAR acquisition time-lag and that the magnitude of change
will vary according the habitat preferences of the study species
{[}figure{]}. \\
\bottomrule()
\end{longtable}

We predicted that (1) SDMs for early successional species will see the
greatest declines in performance due to more rapid amounts of change in
their preferred habitat during the time lag period and models will no
longer be reliable after 5 years lag. (2) Forest generalist species will
see moderate declines in performance as time lag increases (3) The
effect of time lag on SDMs for mature forest species will be smaller due
the the relative stability of their forest habitats, but will still
observe mild decreases in performance as windfall and disease opens new
gaps in the mature forest canopy. For all species, differences in
predictive maps will be negatively correlated with the age of the forest
when LiDAR was acquired.

Our methodological workflow is illustrated in Figure @ref(fig:workflow).
Analyses were done using R statistical software \citep{R-base}. We
developed SDMs following the methodology outlined by Guisan and Thuiller
-\citet{Guisan2005} with bird data collected by the Calling Lake
Fragmentation project.

The study area is \(\approx\) 11,327 ha of boreal mixedwood forests near
Calling Lake, in northern Alberta, Canada (55º14'51'\,' N, 113º28'59'\,'
W) Figure @ref(fig:studyArea)). Calling Lake is located in the Boreal
Central Mixedwood Natural Subregion of Northern Alberta
\citep{Downing2006}. When the point counts were conducted the area
contained black spruce (\emph{Picea mariana}) fens and mixedwood forests
composed of trembling aspen (\emph{Populus tremuloides}, balsam poplar
(\emph{Populus balsamifera}), and white spruce (\emph{Picea glauca}),
with alder (\emph{Alnus spp.}) and willow (\emph{Salix spp.}) dominated
understories \citep{Schmiegelow1997}. Experimental forest harvesting
done through the Calling Lake Fragmentation Project has left a patchwork
of successional stages amidst tracts of unharvested forest
(\citet{Schmiegelow1997}). left a patchwork of forests of different
stages of recovery alongside unharvested tracks
(\citet{Schmiegelow1997}).

Data Sources Point count data from the Calling Lake Fragmentation Study
was accessed through the Boreal Avian Modelling Project's (BAM) avian
database {[}REF{]}. Point count locations have had sixteen years of
consecutive years of point count surveys, providing an opportunity to
test the effects of LiDAR acquisition time lag on SDMs. See Schmiegelow
et al. -\citet{Schmiegelow1997} for details on the Calling Lake
Fragmentation Project's study design. Summer bird surveys were conducted
annually between 1995 and 2014. Point count durations were five minutes.
Within a radius of one hundred meters. Morning point counts
(sunrise-10:00) 3-5 between 16 May and 7 July. We used detection data
from 150 stations that all had consecutive annual surveys between 1993
and 2014. Stations were selected that were within 0 to 16 years of LiDAR
acquisition date (296 stations and \textgreater14000 individual point
counts). Point count locations were spaced \(\approx\) 200 m apart.

To reduce the likelihood of double-counting birds or incorrect
assignment of observations to site type, we restricted analyses to
detections within the 50 m sampling radius.''

We performed our time lag analysis on seven bird species associated with
different nesting and foraging guilds, forest strata, habitat
structures, and forest age classes (TABLE). \citep{EltonTraits2021}.

Species were selected that exhibited low variability in the total number
of detections each year across the 16 years modelled (CV \textless{}
0.5). To ensure that the selected species was abundant enough to model,
we chose species that were detected in \textgreater{} 10\% of all point
count events. Variability of detections between years

The seven selected species included early successional specialists:
mature forests species, habitat generalist \ldots{} American Redstart
(\emph{Setophaga ruticilla}), Black-throated Green Warbler
(\emph{Setophaga virens}), and Swainson's Thrush (\emph{Catharus
ustulatus}), Mourning Warbler (\emph{Geothlypis philadelphia}),
White-throated Sparrow (\emph{Zonotrichia albicollis}), , Winter
Wren(\emph{Troglodytes hiemalis}).

Based on this, we selected Black-throated Green Warbler (BTNW), a mature
forest species; Swainson's Thrush (SWTH) a forest generalist; and
Mourning Warbler (MOWA) an early-seral specialist. YBSA Yellow-bellied
Sapsucker AMRE American redstart (mid seral) WTSP White throated sparrow
(early)

To minimize the influence of forest edges and adjacent differently aged
forest, we excluded point count stations with high variation of forest
age classes within a hundred meter radius (SD \textgreater5 yrs)

We matched point count data from the BAM database with LiDAR data
collected by the provincial government of Alberta (GOA) between 2008 and
2009, classified landcover data from CAS-FRI {[}REF{]}, and climate data
from \_.

LiDAR data covering the study area collected between 2008-2009 was
supplied by Alberta Agriculture and Forestry, Government of Alberta. The
data was part of a wider effort to gather wall-to-wall LiDAR coverage
across the province. For an overview of the LiDAR specifications and
collection protocols see Alberta Environment and Sustainable Resource
Development -\citet{AESRD2013}.

We selected model covariates from 34 lidar candidate predictors
associated with canopy closure, height, and vegetation density. Metrics
were summarized within an 100 meter radius of each point count station
location using the \emph{raster} package in R \citep{R-raster}.

``CAS-FRI is a compilation of standardized forest resource inventory
data from across Canada. It includes common vegetation and disturbance
attributes used in bird habitat models, including forest composition and
disturbance history \citep{Cumming2011a}. We used the following CAS-FRI
attributes as model covariates in our analysis: post-harvest forest type
(coniferous, deciduous, or mixed-wood stand), vegetation composition,
crown closure, canopy height, and harvest intensity.''{[}self{]}
Estimated forest age at the time of the point count, and disturbance
history. As there wasn't much variation in dominant vegetation species
of survey locations, we excluded vegetation composition as a covariate
in our models.

Forest age, forest age class, disturbance history.

Forest age classes adapted from \citet{chen2002dynamics}.

To minimize the influence of forest edges and adjacent differently aged
forest, remove stations that are surrounded by a variety of forest ages
(sd\textgreater5).

We developed SDMs following the methodology outlined by Guisan and
Thuiller -\citet{Guisan2005}. Using point counts conducted the same year
as the LiDAR acquisition date, we used mixed effect logistic regression
to assess the influence of candidate lidar predictors on specifics
occupancy using the \emph{lme4} package in
R\citep{batesFittingLinearMixedeffects2015} using station location as a
random effect. We built global models using all available LiDAR
predictors as fixed effects. We built candidate mixed effect logistic
regression models useing all available lidar predictors n=34. To avoid
multicollinearity between predictors we used Pearson correlation and VIF
scores to iteratively removed highly correlated predictors from our
models. We kept metrics with low correlation (\emph{r} \textless{} 0.5
and VIF \textless{} 3) and are associated with different measures of
vegetation structure: height, cover, and structural complexity
\citep{valbuenaStandardizingEcosystemMorphological2020}. For correlated
metrics associated with similar, we selected the most statistically
significant predictor. Once we had a global model wth a reduced set of
candidate predictors variables, deselected the best model using the
\emph{MuMIn} package in R \citet{bartonMuMInMultimodelInference2020}.

I will combine LiDAR metrics with the CAS-FRI data to model species
habitat relationships for American Redstart, Black-throated Green
Warbler, and Swainson's Thrush. I will evaluate bird abundance as a
function of LiDAR metrics and CAS-FRI data \citep{MacKenzie2006} for
each year (time lag) for 10 years after the LiDAR acquisition date using
GLMs. To accommodate the influence of survey methods and nuisance
parameters on detection probabilities, I will generate statistical
offsets via QPAD \citep{SolymosMatsuoka2013}.

While AUCs dididnt significantly decline for all species, the mapped
spatial distribution did, suggesting that timelag plays a role.

\bibliography{library.bib}


\end{document}
