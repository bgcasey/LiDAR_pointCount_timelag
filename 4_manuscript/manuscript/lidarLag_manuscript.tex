% Options for packages loaded elsewhere
\PassOptionsToPackage{unicode}{hyperref}
\PassOptionsToPackage{hyphens}{url}
\PassOptionsToPackage{dvipsnames,svgnames,x11names}{xcolor}
%
\documentclass[
  12pt,
]{article}
\usepackage{amsmath,amssymb}
\usepackage{lmodern}
\usepackage{setspace}
\usepackage{iftex}
\ifPDFTeX
  \usepackage[T1]{fontenc}
  \usepackage[utf8]{inputenc}
  \usepackage{textcomp} % provide euro and other symbols
\else % if luatex or xetex
  \usepackage{unicode-math}
  \defaultfontfeatures{Scale=MatchLowercase}
  \defaultfontfeatures[\rmfamily]{Ligatures=TeX,Scale=1}
\fi
% Use upquote if available, for straight quotes in verbatim environments
\IfFileExists{upquote.sty}{\usepackage{upquote}}{}
\IfFileExists{microtype.sty}{% use microtype if available
  \usepackage[]{microtype}
  \UseMicrotypeSet[protrusion]{basicmath} % disable protrusion for tt fonts
}{}
\makeatletter
\@ifundefined{KOMAClassName}{% if non-KOMA class
  \IfFileExists{parskip.sty}{%
    \usepackage{parskip}
  }{% else
    \setlength{\parindent}{0pt}
    \setlength{\parskip}{6pt plus 2pt minus 1pt}}
}{% if KOMA class
  \KOMAoptions{parskip=half}}
\makeatother
\usepackage{xcolor}
\usepackage[margin=2.54cm]{geometry}
\usepackage{longtable,booktabs,array}
\usepackage{calc} % for calculating minipage widths
% Correct order of tables after \paragraph or \subparagraph
\usepackage{etoolbox}
\makeatletter
\patchcmd\longtable{\par}{\if@noskipsec\mbox{}\fi\par}{}{}
\makeatother
% Allow footnotes in longtable head/foot
\IfFileExists{footnotehyper.sty}{\usepackage{footnotehyper}}{\usepackage{footnote}}
\makesavenoteenv{longtable}
\usepackage{graphicx}
\makeatletter
\def\maxwidth{\ifdim\Gin@nat@width>\linewidth\linewidth\else\Gin@nat@width\fi}
\def\maxheight{\ifdim\Gin@nat@height>\textheight\textheight\else\Gin@nat@height\fi}
\makeatother
% Scale images if necessary, so that they will not overflow the page
% margins by default, and it is still possible to overwrite the defaults
% using explicit options in \includegraphics[width, height, ...]{}
\setkeys{Gin}{width=\maxwidth,height=\maxheight,keepaspectratio}
% Set default figure placement to htbp
\makeatletter
\def\fps@figure{htbp}
\makeatother
\setlength{\emergencystretch}{3em} % prevent overfull lines
\providecommand{\tightlist}{%
  \setlength{\itemsep}{0pt}\setlength{\parskip}{0pt}}
\setcounter{secnumdepth}{5}
\newlength{\cslhangindent}
\setlength{\cslhangindent}{1.5em}
\newlength{\csllabelwidth}
\setlength{\csllabelwidth}{3em}
\newlength{\cslentryspacingunit} % times entry-spacing
\setlength{\cslentryspacingunit}{\parskip}
\newenvironment{CSLReferences}[2] % #1 hanging-ident, #2 entry spacing
 {% don't indent paragraphs
  \setlength{\parindent}{0pt}
  % turn on hanging indent if param 1 is 1
  \ifodd #1
  \let\oldpar\par
  \def\par{\hangindent=\cslhangindent\oldpar}
  \fi
  % set entry spacing
  \setlength{\parskip}{#2\cslentryspacingunit}
 }%
 {}
\usepackage{calc}
\newcommand{\CSLBlock}[1]{#1\hfill\break}
\newcommand{\CSLLeftMargin}[1]{\parbox[t]{\csllabelwidth}{#1}}
\newcommand{\CSLRightInline}[1]{\parbox[t]{\linewidth - \csllabelwidth}{#1}\break}
\newcommand{\CSLIndent}[1]{\hspace{\cslhangindent}#1}
\usepackage{booktabs}
\usepackage{lineno}
\linenumbers
\usepackage {hyperref}
\hypersetup {colorlinks = true, linkcolor = blue, urlcolor = blue}
\usepackage{multirow}
\usepackage{amsmath, amssymb} 
\usepackage{enumitem, array}

% set spacing
\usepackage{setspace}\doublespacing

\ifLuaTeX
  \usepackage{selnolig}  % disable illegal ligatures
\fi
\IfFileExists{bookmark.sty}{\usepackage{bookmark}}{\usepackage{hyperref}}
\IfFileExists{xurl.sty}{\usepackage{xurl}}{} % add URL line breaks if available
\urlstyle{same} % disable monospaced font for URLs
\hypersetup{
  pdftitle={The influence of time-lag between LiDAR and wildlife survey data on species distribution models},
  pdfauthor={Brendan Casey},
  pdfkeywords={one, two, three},
  colorlinks=true,
  linkcolor={Maroon},
  filecolor={Maroon},
  citecolor={blue},
  urlcolor={Blue},
  pdfcreator={LaTeX via pandoc}}

\title{The influence of time-lag between LiDAR and wildlife survey data on species distribution models}
\author{Brendan Casey}
\date{2023-05-09}

\begin{document}
\maketitle
\begin{abstract}
LiDAR has the potential to improve bird-habitat models by providing high resolution structural covariates which can give insight into bird-habitat relationships. However, LiDAR acquisitions do not always coincide in time with point count surveys. It is unclear how much this temporal misalignment influence the predictive accuracy of bird-habitat models that use LiDAR derived predictor variables. As disturbance-succession cycles change vegetation structure, eventually LiDAR metrics will no longer reflect ground conditions. Thus the usefulness of LiDAR explanatory variables will degrade. Here, we evaluated how time-lag between LiDAR acquisitions and bird surveys influenced model robustness for early-successional, mature-forest, and forest generalist birds. We found that for species occupying older, more stable forests, a time difference of up to 15 years has only a small impact on the predictive power of LiDAR based bird-habitat models. However, for early-successional birds, our findings suggest that a time difference of 5-13 years between LIDAR and bird data may reduce model performance.
\end{abstract}

{
\hypersetup{linkcolor=}
\setcounter{tocdepth}{2}
\tableofcontents
}
\listoffigures
\listoftables
\setstretch{1.25}
\doublespacing

\hypertarget{introduction}{%
\section{Introduction}\label{introduction}}

The composition and structure of forests are changing in response to climate change, shifts to natural disturbance regimes, and increasing industrial development (\protect\hyperlink{ref-Brandt2013}{Brandt et al., 2013}). Predictive models linking field observations to environmental variables can reveal how birds respond to these changes (\protect\hyperlink{ref-Carrillo-Rubio2014}{Carrillo-Rubio et al., 2014}; \protect\hyperlink{ref-englerAvianSDMsCurrent2017}{Engler et al., 2017}; \protect\hyperlink{ref-guisanPredictiveHabitatDistribution2000}{Guisan and Zimmermann, 2000}; \protect\hyperlink{ref-He2015}{He et al., 2015}). Broadly known as species distribution models (SDMs), this family of statistical methods predict bird distributions by comparing habitat where individuals were observed against habitat where they were absent (\protect\hyperlink{ref-Guisan2005}{Guisan and Thuiller, 2005}). SDMs and resulting predictive distribution maps are used to understand bird habitat preferences and the drivers of broad scale population declines and have applications in conservation management planning and environmental impact assessments (\protect\hyperlink{ref-englerAvianSDMsCurrent2017}{Engler et al., 2017}; \protect\hyperlink{ref-franklinMappingSpeciesDistributions2010}{Franklin, 2010}).

Many factors influence the predictive capacity of SDMs, but the inclusion of ecologically relevant spatial covariates are key drivers of model accuracy (\protect\hyperlink{ref-fourcadePaintingsPredictDistribution2018}{Fourcade et al., 2018}; \protect\hyperlink{ref-Franklin1995}{Franklin, 1995}; \protect\hyperlink{ref-Vaughn2003}{Vaughn and Ormerod, 2003}). Bird SDMs often rely on categorical predictors derived from digital maps delineating land cover, vegetation composition, and human footprint. While useful, they often miss key forest features driving habitat selection, namely those related to vegetation structure.

Vegetation structure influences the abundance, distribution, and behavior of birds (\protect\hyperlink{ref-Davies2014a}{Davies and Asner, 2014}; \protect\hyperlink{ref-MacArthurMacArthur1961}{MacArthur and MacArthur, 1961}). The height and density of vegetation influence where birds perch, feed, and reproduce (\protect\hyperlink{ref-Bradbury2005}{Bradbury et al., 2005}) by mediating microclimates, providing shelter from weather (\protect\hyperlink{ref-CarrascalDiaz2006}{Carrascal and Diaz, 2006}), concealment from predators (\protect\hyperlink{ref-GotmarkBlomqvist1995}{Gotmark et al., 1995}), and creating habitat for insect prey (\protect\hyperlink{ref-halajImportanceHabitatStructure2000}{Halaj et al., 2000}). Light Detection and Ranging (LiDAR) can characterize these three-dimensional forest structures (\protect\hyperlink{ref-limLiDARRemoteSensing2003}{Lim et al., 2003}). Common LiDAR derived metrics correspond with vegetation height, cover, structural complexity, and density of forest strata (\protect\hyperlink{ref-Bae2018}{Bae et al., 2018}; \protect\hyperlink{ref-Davies2014a}{Davies and Asner, 2014}; \protect\hyperlink{ref-Kortmann2018}{Kortmann et al., 2018}; \protect\hyperlink{ref-Lefsky2002}{Lefsky et al., 2002}; \protect\hyperlink{ref-Renner2018}{Renner et al., 2018}). Used as predictor variables, LiDAR metrics can improve the predictive power of bird SDMs (\protect\hyperlink{ref-Bae2014}{Bae et al., 2014}; \protect\hyperlink{ref-clawgesUseAirborneLidar2008}{Clawges et al., 2008}; \protect\hyperlink{ref-farrellUsingLiDARderivedVegetation2013b}{Farrell et al., 2013}; \protect\hyperlink{ref-ficetolaHowManyPredictors2014}{Ficetola et al., 2014}).

Publicly funded regional LIDAR data and space-based sensors like NASA's Ice, Cloud and Land Elevation Satellite-2 (ICESat-2) and Global Ecosystem Dynamics Investigation (GEDI), have made large amounts of wall-to-wall structural data available to researchers (\protect\hyperlink{ref-abdalatiICESat2LaserAltimetry2010}{Abdalati et al., 2010}; \protect\hyperlink{ref-coopsForestStructureHabitat2016}{Coops et al., 2016}; \protect\hyperlink{ref-dubayahGlobalEcosystemDynamics2020a}{Dubayah et al., 2020}). However, LiDAR continues to be under-used in bird ecology. The limited temporal resolution of most LiDAR products may be a factor. LiDAR is often limited to a single season, with long multiyear gaps between repeat surveys. Temporal misalignment between wildlife surveys and LiDAR is common.

Temporal misalignment occurs when wildlife surveys and LiDAR acquisitions are done at different times (\protect\hyperlink{ref-babcockModelingForestBiomass2016}{Babcock et al., 2016}; \protect\hyperlink{ref-VierlingSwift2014}{Vierling et al., 2014}). It's unclear how much temporal misalignment influences the performance of LiDAR based SDMs. Disturbance-succession cycles drive changes in vegetation structure, and eventually, LiDAR gathered over a season will no longer reflect ground conditions. This can occur when the surveyed forest transitions between stages of stand development, e.g.~from stand initiation to stem exclusion (\protect\hyperlink{ref-babcockModelingForestBiomass2016}{Babcock et al., 2016}; \protect\hyperlink{ref-brassardStandStructureComposition2010}{Brassard and Chen, 2010}). Thus, temporal misalignment can impact the power of bird SDMs as successional changes in forest structure influence habitat selection by birds (\protect\hyperlink{ref-sittersAssociationsOccupancyHabitat2014}{Sitters et al., 2014}).

Consider Canada's boreal forests. It is a dynamic successional mosaic driven by forestry, fire, and energy exploration (\protect\hyperlink{ref-Brandt2013}{Brandt et al., 2013}; \protect\hyperlink{ref-gauthierBorealForestHealth2015}{Gauthier et al., 2015}). The landscape is a patchwork of early to late-successional stands with distinct structural characteristics (\protect\hyperlink{ref-Bergeron2012}{Bergeron and Fenton, 2012}; \protect\hyperlink{ref-brassardStandStructureComposition2010}{Brassard and Chen, 2010}) and bird communities (\protect\hyperlink{ref-Schieck2006}{Schieck and Song, 2006}). In early-successional forests, bird communities are dominated by species that nest and forage in open vegetation, wetlands, and shrubs, along with some habitat generalists. As trees regenerate and the stand's structural properties change, open habitat species give way to species associated with closed canopy forests that vary in underlying vertical structure over time (\protect\hyperlink{ref-lestonLongtermChangesBoreal2018}{Leston et al., 2018}; \protect\hyperlink{ref-Schieck2006}{Schieck and Song, 2006}).

Thus, succession occurring between LiDAR and wildlife surveys may influence SDM performance. Consequently, LiDAR's usefulness as a source of explanatory variables can degrade as temporal misalignment increases. For researchers pairing LiDAR covariates with long-term wildlife survey data, this can lead to a trade-off: (1) minimize temporal misalignment by reducing the sample size to survey data gathered near the time of the LiDAR acquisition, or (2) maximize sample size and risk sacrificing model power.

To inform this trade-off, we addressed the question of how much temporal misalignment is acceptable in LiDAR based SDMs. Our objectives were to (1) evaluate how the time lag between LiDAR acquisitions and bird surveys influence the performance of SDMs across a gradient of 0 to 15 years, (2) compare the influence of temporal misalignment on models for early-successional, mid-successional, mature-forest, and forest generalist birds, and (3) assess how differences in resultant predictive distribution maps correlate with forest age.

The effects of temporal misalignment on SDMs will likely vary by habitat type (e.g.~forest age, disturbance history, and dominant vegetation) and the life history characteristics of the study species. We predicted that the performance of SDMs will decrease with increased temporal misalignment and that the magnitude of change will vary according to the habitat associations of the focal species. We predicted that (1) SDMs for early-successional associates, Mourning Warbler (\emph{Geothlypis philadelphia}) and White-throated Sparrow (\emph{Zonotrichia albicollis}), would be most affected by temporal misalignment because of faster vertical growth rates of establishment trees and loss of dense shrub layers (\protect\hyperlink{ref-fallsWhitethroatedSparrowZonotrichia2020}{Falls and Kopachena, 2020}; \protect\hyperlink{ref-mccarthy2001gap}{McCarthy, 2001}; \protect\hyperlink{ref-pitocchelliMourningWarblerGeothlypis2020}{Pitocchelli, 2020}). (2) SDMs for mid-seral species like American Redstart (\emph{Setophaga ruticilla}) that are associated with dense midstory vegetation, would see moderate declines in performance as temporal misalignment increases due to self-thinning during the stem exclusion stage of succession (\protect\hyperlink{ref-brassardStandStructureComposition2010}{Brassard and Chen, 2010}; \protect\hyperlink{ref-sherryAmericanRedstartSetophaga2020a}{Sherry et al., 2020}). And (3) mature forest associates, Black-throated Green Warbler (\emph{Setophaga virens}), will be least effected by temporal misalignment as the processes effecting mature forest canopy structure (insect defoliation and windthrow) happen at too small a scale to effect overall model performance (\protect\hyperlink{ref-morseBlackthroatedGreenWarbler2020a}{Morse and Poole, 2020}; \protect\hyperlink{ref-VierlingSwift2014}{Vierling et al., 2014}). For all species, we predicted that differences in distribution maps will be negatively correlated with forest age.

\hypertarget{methods}{%
\section{Methods}\label{methods}}

Our methodological workflow is illustrated in Figure \ref{fig:2workflow}. Analyses were done using R statistical software (\protect\hyperlink{ref-R-base}{R Core Team, 2020}). We built SDMs using bird data from the Calling Lake Fragmentation project (\protect\hyperlink{ref-Schmiegelow1997}{Schmiegelow et al., 1997}).

\begin{figure}[htb]
\includegraphics[width=1\linewidth,]{../3_output/figures/chapter1_workflow} \caption{Conceptual diagram of our methodology. SDM methods were repeated at every time lag for each species. SDMs were compared using AUC and correlation between predictive maps.}\label{fig:2workflow}
\end{figure}

\hypertarget{study-area}{%
\subsection{Study area}\label{study-area}}

We used bird survey data from the Calling Lake Fragmentation Experiment (\protect\hyperlink{ref-Schmiegelow1997}{Schmiegelow et al., 1997}). Surveys were conducted across \(\approx\) 14,000 ha of boreal mixedwood forests near Calling Lake, in northern Alberta, Canada (55º14'51'\,' N, 113º28'59'\,' W) (Figure \ref{fig:2studyArea}). The experiment was designed to study the long-term impacts of forest harvesting on birds (\protect\hyperlink{ref-hannonCorridorsMayNot2002}{Hannon and Schmiegelow, 2002}; \protect\hyperlink{ref-lestonLongtermChangesBoreal2018}{Leston et al., 2018}; \protect\hyperlink{ref-Schmiegelow1997}{Schmiegelow et al., 1997}). The study's experimental harvest treatments have led to a landscape patchwork of early- to mid- successional stands surrounded by tracts of unharvested mature forests. When the experiment began in 1994, the landscape was dominated by older mixedwood forests composed of trembling aspen (\emph{Populus tremuloides}, balsam poplar (\emph{Populus balsamifera}) and white spruce (\emph{Picea glauca}) and treed bogs containing black spruce (\emph{Picea mariana}) and larch (\emph{Larix laricina}). Understory vegetation in the mixedwood forests was composed mostly of alder (\emph{Alnus spp.}) and willow species(\emph{Salix spp.}).


\begin{figure}[htb]
\includegraphics[width=1\linewidth,]{../3_output/maps/studyArea_inset} \caption{Locations of point count survey sites from the Calling Lake Fragmentation Study near Calling Lake, Alberta (\protect\hyperlink{ref-Schmiegelow1997}{Schmiegelow et al., 1997}). Repeat point counts were conducted during the breeding seasons from 1993 and 2015.}\label{fig:2studyArea}
\end{figure}

\hypertarget{bird-data}{%
\subsection{Bird data}\label{bird-data}}

The Calling Lake Fragmentation Experiment included long term bird monitoring via annual repeated point counts. Point counts were done for 20 consecutive breeding seasons (from 1995-2015). As the experiment's study area overlaps spatially with government wall-to-wall LiDAR coverage, there is an opportunity to study the impacts of temporal misalignment between point counts and LiDAR on bird SDMs.

We used detection data from 187 stations where consecutive annual point counts were conducted within sixteen years of the LiDAR acquisition date. Stations were spaced \(\approx\) 200 m apart. At each station, three to five morning point count surveys were conducted over each breeding season (May 16 to July 7) between sunrise and 10:00 h. Observers recorded the species detected during each five minute point count interval within sampling radii of 50 and 100 meters. See Schmiegelow et al. (\protect\hyperlink{ref-Schmiegelow1997}{1997}) for further information on the Calling Lake Fragmentation Experiment's study design and point count protocols. To minimize the influence of forest edges on model predictions, we limited point count stations to those conducted within a single forest stand age (\emph{SD \textless5 yrs} within a hundred meter buffer of the station). We accessed point count data using the Boreal Avian Modelling project's avian database (\protect\hyperlink{ref-BAM2018}{Boreal Avian Modelling Project (BAM), 2018}).

We tested the effects of LiDAR temporal misalignment on six bird species common to the study area (detected in \(\ge\) 10\% of all point count events) that were associated with different forest age classes: American Redstart (\emph{Setophaga ruticilla}), Black-throated Green Warbler (\emph{Setophaga virens}), Mourning Warbler (\emph{Geothlypis philadelphia}), Swainson's Thrush (\emph{Catharus ustulatus}), White-throated Sparrow (\emph{Zonotrichia albicollis}), and Winter Wren (\emph{Troglodytes hiemalis}). The focal species showed low variability in the total number of detections each year across the 16 years modelled (\emph{CV \textless{} 0.5}).

\hypertarget{predictor-variables}{%
\subsection{Predictor variables}\label{predictor-variables}}

Habitat covariates included LiDAR vegetation metrics provided by the provincial government of Alberta (GOA), forest stand attributes from the Common Attribute Schema for Forest Resource Inventories (CAS-FRI) (\protect\hyperlink{ref-Cumming2011a}{Cosco, 2011}), and mean summer NDVI calculated from a time series of Landsat images (\protect\hyperlink{ref-geologicalsurveyLandsat47Surface2018}{Survey, 2018}) (Table \ref{tab:2covDes}).

\begin{table}[h!]

\caption{\label{tab:2covDes}Spatial covariates included in the analysis.}
\centering
\resizebox{\linewidth}{!}{
\fontsize{9}{11}\selectfont
\begin{tabular}[t]{llr}
\toprule
Metric & Source & Description\\
\midrule
elev\textunderscore mean & LiDAR & Mean height\\
elev\textunderscore maximum & LiDAR & Maximum height\\
elev\textunderscore cv & LiDAR & Height coefficient of variation\\
canopy\textunderscore relief\textunderscore ratio & LiDAR & Canopy relief ratio (mean - min)/(max-min)\\
elev\textunderscore p50 & LiDAR & 50th percentile of canopy height\\
\addlinespace
elev\textunderscore kurtosis & LiDAR & Height kurtosis\\
elev\textunderscore p99 & LiDAR & 99th percentile of canopy height\\
elev\textunderscore stddev & LiDAR & Height standard deviation\\
percentage\textunderscore first\textunderscore returns\textunderscore above\textunderscore 2pnt00 & LiDAR & Percentage of first returns above 2 m\\
percentage\textunderscore first\textunderscore returns\textunderscore above\textunderscore mean & LiDAR & Percentage of first returns above the mean return height\\
\addlinespace
total\textunderscore all\textunderscore returns & LiDAR & Total all returns\\
elev\textunderscore p95 & LiDAR & 95th percentile of canopy height\\
strata\textunderscore 0pnt15\textunderscore to\textunderscore 2pnt00 & LiDAR & Proportion of points between 0.15 and 2 m\\
strata\textunderscore 2pnt00\textunderscore to\textunderscore 4pnt00 & LiDAR & Proportion of points between 2 and 4 m\\
strata\textunderscore 4pnt00\textunderscore to\textunderscore 6pnt00 & LiDAR & Proportion of points between 4 and 6 m\\
\addlinespace
strata\textunderscore 6pnt00\textunderscore to\textunderscore 8pnt00 & LiDAR & Proportion of points between 6 and 8 m\\
strata\textunderscore 8pnt00\textunderscore to\textunderscore 10pnt00 & LiDAR & Proportion of points between 8 and 10 m\\
strata\textunderscore 10pnt00\textunderscore to\textunderscore 15pnt00 & LiDAR & Proportion of points between 10 and 15 m\\
strata\textunderscore 15pnt00\textunderscore to\textunderscore 20pnt00 & LiDAR & Proportion of points between 15 and 20 m\\
strata\textunderscore 20pnt00\textunderscore to\textunderscore 25pnt00 & LiDAR & Proportion of points between 20 and 25 m\\
\addlinespace
strata\textunderscore 25pnt00\textunderscore to\textunderscore 30pnt00 & LiDAR & Proportion of points between 25 and 30 m\\
strata\textunderscore 30pnt00\textunderscore to\textunderscore 50pnt00 & LiDAR & Proportion of points between 30 and 50 m\\
forest\textunderscore age & CAS-FRI & Mean age of the forest stand\\
NDVI & Landsat 5, 7, and 8 & Mean NDVI\\
\bottomrule
\end{tabular}}
\end{table}

LiDAR data covering the study area collected between 2008-2009 was supplied by Alberta Agriculture and Forestry, Government of Alberta.
Airborne LiDAR was gathered between 2008-2009 by Alberta Agriculture and Forestry, Government of Alberta. The data was part of a larger provincial wall-to-wall LiDAR mapping effort. For an overview of the LiDAR specifications and collection protocols see Alberta Environment and Sustainable Resource Development -Alberta Environment and Sustainable Resource Development (\protect\hyperlink{ref-AESRD2013}{2013}).
The Government of Alberta provided us with 30m LiDAR raster layers representing vegetation height, cover, and density metrics. The rasters were calculated from point cloud data using FUSION software (\protect\hyperlink{ref-mcgaugheyFUSIONLDVSoftware2018}{McGaughey, 2018}). For each raster, we calculated the mean pixel value within a 100 meter radius of point count stations using the \emph{raster} package in R (\protect\hyperlink{ref-R-raster}{Hijmans, 2020}).

Forest stand attributes were extracted from the Common Attribute Schema for Forest Resource Inventories (CAS-FRI). CAS-FRI is a standardized collection of 2 ha forest inventory geospatial data (\protect\hyperlink{ref-Cumming2011a}{Cosco, 2011}). CAS-FRI stand attributes were interpreted using 1:10,000 to 1:40,000 aerial photography flown between 1987 and 2010. For each point-count station location we determined the disturbance history and mean forest age. As there wasn't much variation in the dominant vegetation species at survey locations, we excluded vegetation composition as a covariate in our models.

We used the Normalized Difference Vegetation Index (NDVI) as an indicator of vegetation cover (\protect\hyperlink{ref-pettorelliNormalizedDifferenceVegetation2011}{Pettorelli et al., 2011}). We generated 30 m composite NDVI rasters from 1995 to 2015 using surface reflectance imagery from the Landsat 5 Thematic Mapper (bands 3 and 4), the Landsat 7 Enhanced Thematic Mapper (bands 3 and 4), and the Landsat 8 Operational Land Imager (bands 4 and 5) (\protect\hyperlink{ref-geologicalsurveyLandsat47Surface2018}{Survey, 2018}). Satellite images were accessed and processed using the Google Earth Engine (GEE) Code Editor (\protect\hyperlink{ref-gorelickGoogleEarthEngine2017}{Gorelick et al., 2017}). As all of the point counts occurred during the summer breeding season, we limited Landsat images to those taken between June and September. Images were masked to exclude snow, cloud, and cloud shadow pixels using the CFMask algorithm (\protect\hyperlink{ref-fogaCloudDetectionAlgorithm2017}{Foga et al., 2017}). We generated annual median composites of masked Landsat images and calculated NDVI rasters from the composites (\(NDVI=\frac{NIR-R}{NIR+R}\)) (\protect\hyperlink{ref-USGS_NDVI}{USGS, n.d.}). For each survey year we calculated the mean values of NDVI pixels within a 100 m buffer of point count locations.

\hypertarget{analyses}{%
\subsection{Analyses}\label{analyses}}

We evaluated the effects of LiDAR temporal misalignment on model performance by comparing mixed effects logistic regression models. We built models using the \texttt{glmer} function in the R package \emph{lme4} (\protect\hyperlink{ref-batesFittingLinearMixedeffects2015}{Bates et al., 2015}). To accommodate the influence of survey methods and nuisance parameters on detection probabilities, we included statistical offsets in the models generated using QPAD (\protect\hyperlink{ref-SolymosMatsuoka2013}{Sólymos et al., 2013}).

We used the following multi-step process for each focal species. In Step 1, we grouped detection data according to the amount of temporal misalignment with LiDAR. There were 16 groups, one group for each year of time-lag between LiDAR and point count surveys (zero through fifteen years).

In Step 2, we built and evaluated models for the zero time-lag group of point counts. We first computed a global model with all candidate predictor variables as fixed effects and station location as a random effect. We checked for nonlinear relationships between response and predictors by separately evaluating, linear, quadratic, and cubic functions of each variable. To avoid multicollinearity between predictors, we used Pearson correlation coefficients and VIF scores to iteratively remove highly correlated predictors from the global model. We kept metrics with low correlation (\emph{r} \textless{} 0.5 and VIF \textless{} 3) that were associated with different vegetation structure categories: height, cover, and complexity (\protect\hyperlink{ref-valbuenaStandardizingEcosystemMorphological2020}{Valbuena et al., 2020}). For correlated metrics associated with the same category, we selected the variable with the lowest \emph{P} value. We evaluated models consisting of the remaining predictors using the `dredge' function in the R package \emph{MuMIn} (\protect\hyperlink{ref-bartonMuMInMultimodelInference2020}{Bartoń, 2020}). We defined the top model as that with the lowest Akaike's Information Criterion (AIC)(\protect\hyperlink{ref-burnhamModelSelectionMultimodel2002}{Burnham and Anderson, 2002}). We calculated pseudo-\(R^2\) as a measure of explanatory power Nakagawa and Schielzeth (\protect\hyperlink{ref-nakagawaGeneralSimpleMethod2013}{2013}){]}. For models with similar AIC values (a difference less than two) we selected the model with the largest pseudo-\(R^2\).

In Step 3, we applied the Step 2 model to the remaining groups of time-lag point counts. For each group, we used the fitted model and a raster stack of covariates to map species occurrence probability using the `predict' function in the R package \emph{raster} (\protect\hyperlink{ref-hijmansRasterGeographicData2021}{Hijmans, 2021}).

In Step 4, we compared the performance of different time-lag models. We compared their predictive accuracy using the area under the receiver operating curve (ROC) (AUC) calculated using the `auc' function in the pROC package in R (\protect\hyperlink{ref-robinPROCOpensourcePackage2011}{Robin et al., 2011}). Models with an AUC \textgreater0.7 were considered moderately predictive of species occurrence (\protect\hyperlink{ref-vanagasReceiverOperatingCharacteristic2004}{Vanagas, 2004}) We tracked the amount of LiDAR time-lag necessary for models to have an AUC \textless0.70. The contribution of individual fixed effects were estimated by calculating semi-partial \(R^2\) values using the \emph{r2beta} function from the \emph{r2glmm} package using standardized general variances (\protect\hyperlink{ref-jaegerR2glmmComputesSquared2017}{Jaeger, 2017}). We compared the predictive maps for different time-lag groups by calculating the per pixel differences between them. I.e., we subtracted the zero time lag map from the map of each subsequent time lag group, resulting in 15 ``difference'' rasters. We used Pearson's correlations to examine the relationship between differences in species occurrence probability and forest age.

\hypertarget{results}{%
\section{Results}\label{results}}

The LiDAR variables used in top performing models varied by species (Table \ref{tab:2topModels}). The most common predictor variables across species were maximum vegetation height (used in all top models), and mean summer NDVI (used in top models for five out of six species). The rates and magnitude by which AUC was effected by LiDAR-bird survey temporal misalignment varied by species (Figure \ref{fig:2AUCLag}).

\renewcommand{\arraystretch}{2}
\begin{table}[h!]

\caption{\label{tab:2topModels}The fixed effects and summary statistics for top models for each species.}
\centering
\resizebox{\linewidth}{!}{
\fontsize{8}{10}\selectfont
\begin{tabular}[t]{>{}l>{\raggedright\arraybackslash}p{25em}rrr}
\toprule
Species & Fixed effects & $r^{2}m$ & $r^{2}c$ & AUC\\
\midrule
American Redstart &  &  &  & \\

(\emph{Setophaga ruticilla}) & \multirow[t]{-2}{25em}{\raggedright\arraybackslash ndvi +  elev\textunderscore 2pnt00\textunderscore to\textunderscore 4pnt00\textunderscore return\textunderscore proportion +  elev\textunderscore cv+ elev\textunderscore maximum +  elev\textunderscore p50} & \multirow[t]{-2}{*}{\raggedleft\arraybackslash 0.46} & \multirow[t]{-2}{*}{\raggedleft\arraybackslash 0.62} & \multirow[t]{-2}{*}{\raggedleft\arraybackslash 0.78}\\
\addlinespace
Black-throated Green Warbler &  &  &  & \\

(\emph{Setophaga virens}) & \multirow[t]{-2}{25em}{\raggedright\arraybackslash elev\textunderscore p50 + elev\textunderscore maximum + total\textunderscore all\textunderscore returns} & \multirow[t]{-2}{*}{\raggedleft\arraybackslash 0.48} & \multirow[t]{-2}{*}{\raggedleft\arraybackslash 0.63} & \multirow[t]{-2}{*}{\raggedleft\arraybackslash 0.80}\\
\addlinespace
Mourning Warbler &  &  &  & \\

(\emph{Geothlypis philadelphia}) & \multirow[t]{-2}{25em}{\raggedright\arraybackslash ndvi\textsuperscript{2} +  elev\textunderscore 0pnt15\textunderscore to\textunderscore 2pnt00\textunderscore return\textunderscore proportion+ elev\textunderscore 2pnt00\textunderscore to\textunderscore 4pnt00\textunderscore return\textunderscore proportion + elev\textunderscore maximum *elev\textunderscore stddev+   percentage\textunderscore first\textunderscore returns\textunderscore above} & \multirow[t]{-2}{*}{\raggedleft\arraybackslash 0.49} & \multirow[t]{-2}{*}{\raggedleft\arraybackslash 0.50} & \multirow[t]{-2}{*}{\raggedleft\arraybackslash 0.78}\\
\addlinespace
Swainson's Thrush &  &  &  & \\

(\emph{Catharus ustulatus}) & \multirow[t]{-2}{25em}{\raggedright\arraybackslash ndvi\textsuperscript{2} +  elev\textunderscore maximum + elev\textunderscore p50} & \multirow[t]{-2}{*}{\raggedleft\arraybackslash 0.13} & \multirow[t]{-2}{*}{\raggedleft\arraybackslash 0.16} & \multirow[t]{-2}{*}{\raggedleft\arraybackslash 0.67}\\
\addlinespace
White-throated Sparrow &  &  &  & \\

(\emph{Zonotrichia albicollis}) & \multirow[t]{-2}{25em}{\raggedright\arraybackslash ndvi  + elev\textunderscore cv +  elev\textunderscore maximum + percentage\textunderscore first\textunderscore returns\textunderscore above\textunderscore 2pnt00} & \multirow[t]{-2}{*}{\raggedleft\arraybackslash 0.20} & \multirow[t]{-2}{*}{\raggedleft\arraybackslash 0.33} & \multirow[t]{-2}{*}{\raggedleft\arraybackslash 0.70}\\
\addlinespace
Winter Wren &  &  &  & \\

(\emph{Troglodytes hiemalis}) & \multirow[t]{-2}{25em}{\raggedright\arraybackslash ndvi +  canopy\textunderscore relief\textunderscore ratio+ elev\textunderscore 4pnt00\textunderscore to\textunderscore 6pnt00\textunderscore return\textunderscore proportion+ elev\textunderscore maximum +   percentage\textunderscore first\textunderscore returns\textunderscore above} & \multirow[t]{-2}{*}{\raggedleft\arraybackslash 0.42} & \multirow[t]{-2}{*}{\raggedleft\arraybackslash 0.45} & \multirow[t]{-2}{*}{\raggedleft\arraybackslash 0.70}\\
\bottomrule
\end{tabular}}
\end{table}

\begin{figure}[htb]
\includegraphics[width=1\linewidth,]{../3_output/figures/timelag_stats/sppAll_AUC_lm} \caption{Plot showing the relationship between model AUC and LiDAR temporal misalignment with bird surveys for each species.}\label{fig:2AUCLag}
\end{figure}

\hypertarget{american-redstart}{%
\subsection{American Redstart}\label{american-redstart}}

Occupancy probability for American Redstart increased with NDVI, the coefficient of variation of vegetation height, and the 50th percentile vegetation height, and decreased with maximum vegetation height, and the proportion of lidar returns between 2 and 4 meters high. The model built using temporally aligned covariates explained 46\% of the variance in American redstart occupancy and had an AUC of 0.784. The coefficient of variation of vegetation height contributed most to predicting occupancy (\emph{b} = 3.149, SE = 0.691, \emph{p} \textless{} 0.001, semi-partial \(R^2\) = 0.065), followed by NDVI (\emph{b} = 0.92, SE = 0.267, \emph{p} \textless{} 0.001, semi-partial \(R^2\) = 0.051), and maximum elevation (\emph{b} = -3.633, SE = 0.861, \emph{p} \textless{} 0.001, semi-partial \(R^2\) = 0.046) (Table \ref{tab:2varImp}). The percentage of explained variance did not decline with temporally misaligned LiDAR. For American Redstart, temporal misalignment did not lead to a decrease in model performance as measured by AUC.
\renewcommand{\arraystretch}{1}

\begin{table}

\caption[Predictor variables ranked according to their semi-partial $R^2$.]{\label{tab:2varImp}Predictor variables ranked according to their semi-partial $R^2$. AMRE=American Redstart (\emph{Setophaga ruticilla}); BTNW=Black-throated Green Warbler (\emph{Setophaga virens}); MOWA=Mourning Warbler (\emph{Geothlypis philadelphia}); SWTH=Swainson's Thrush (\emph{Catharus ustulatus}); WIWR=Winter Wren (\emph{Troglodytes hiemalis}); WTSP=White-throated Sparrow (\emph{Zonotrichia albicollis}).}
\centering
\resizebox{\linewidth}{!}{
\fontsize{8}{10}\selectfont
\begin{tabular}[t]{>{\raggedright\arraybackslash}p{20em}rrrrrr}
\toprule
Predictor & AMRE & BTNW & MOWA & SWTH & WIWR & WTSP\\
\midrule
strata\textunderscore 2pnt00\textunderscore to\textunderscore 4pnt00 & 4 &  & 3 &  &  & \\
elev\textunderscore cv & 1 &  &  &  &  & 4\\
elev\textunderscore maximum & 3 & 1 & 8 & 1 & 1 & 1\\
elev\textunderscore p50 & 4 & 2 &  & 2 &  & \\
ndvi\textunderscore lag\textunderscore 0 & 2 &  &  &  & 5 & 2\\
\addlinespace
total\textunderscore all\textunderscore returns &  & 3 &  &  &  & \\
strata\textunderscore 0pnt15\textunderscore to\textunderscore 2pnt00 &  &  & 2 &  &  & \\
elev\textunderscore maximum:elev\textunderscore stddev &  &  & 5 &  &  & \\
elev\textunderscore stddev &  &  & 6 &  &  & \\
percentage\textunderscore first\textunderscore returns\textunderscore above\textunderscore mean &  &  & 4 &  & 4 & \\
\addlinespace
poly(ndvi\textunderscore lag\textunderscore 0, 2)1 &  &  & 1 & 3 &  & \\
poly(ndvi\textunderscore lag\textunderscore 0, 2)2 &  &  & 7 & 4 &  & \\
canopy\textunderscore relief\textunderscore ratio &  &  &  &  & 2 & \\
strata\textunderscore 4pnt00\textunderscore to\textunderscore 6pnt00 &  &  &  &  & 3 & \\
percentage\textunderscore first\textunderscore returns\textunderscore above\textunderscore 2pnt00 &  &  &  &  &  & 3\\
\bottomrule
\end{tabular}}
\end{table}

\hypertarget{black-throated-green-warbler}{%
\subsection{Black-Throated Green Warbler}\label{black-throated-green-warbler}}

Occupancy probability for Black-Throated Green Warbler increased with maximum vegetation height and decreased with the 50th percentile of canopy height and the total of all LiDAR height returns. The model built using temporally aligned covariates explained 48\% of the variance in Black-Throated Green Warbler occupancy and had an AUC of 0.799. Maximum vegetation height contributed the most to model performance (\emph{b} = 2.743, SE = 0.559, \emph{p} \textless{} 0.001, semi-partial \(R^2\) = 0.138) followed by the 50th percentile of canopy height (\emph{b} = -1.439, SE = 0.351, \emph{p} \textless{} 0.001, semi-partial \(R^2\) = 0.035). There was no discernible trend in the explained variance with increasing LiDAR temporal misalignment. However, we observed a significant decrease in model performance (\(R^2\)=.28, \emph{p}\textless0.05). The AUC statistic for the zero-time lag model was less than 0.7 with 14 years of LiDAR-bird survey time lag.

\hypertarget{mourning-warbler}{%
\subsection{Mourning Warbler}\label{mourning-warbler}}

Mourning Warbler occupancy responded positively to the percentage of first returns above the mean vegetation height, NDVI, the density of vegetation below two meters, and maximum vegetation height at low standard deviation of vegetation height. Mourning Warbler occupancy decreased with increased vegetation density between two and four meters. The model built using temporally aligned covariates explained 49\% of the variance in Mourning Warbler occupancy and had an AUC of 0.782. NDVI contributed the most to model predictions (\emph{b} = 25.81, SE = 0.354, \emph{p} \textless{} 0.001, semi-partial \(R^2\) = 0.074)followed by the proportion of vegetation returns below two meters (\emph{b} = 0.797, SE = 0.215, \emph{p} \textless{} 0.001, semi-partial \(R^2\) = 0.028). For Mourning Warbler, we found that increased LiDAR-point count temporal misalignment led to reductions in the amount of explained variance (\(r^2\)=.29, \emph{p}\textless.05) and model performance (\(r^2\)=.29, \emph{p}\textless.05). AUC statistics were less than 0.7 with \textgreater{} 13 years of LiDAR temporal misalignment.

\hypertarget{swainsons-thrush}{%
\subsection{Swainson's Thrush}\label{swainsons-thrush}}

Swainson's Thrush occupancy probability responded non-linearly to NDVI (the probability of occupancy increased with increasing low NDVI values and decreased with higher values). Occupancy probability responded negatively to the 50th percentile of vegetation returns and positively to the maximum vegetation height. The model built using temporally aligned covariates explained 13\% of the variance in Swainson's Thrush occupancy and had an AUC of 0.668. The maximum vegetation height contributed the most to model performance (\emph{b} = 0.955, SE = 0.163, \emph{p} \textless{} 0.001, semi-partial \(R^2\) = 0.067) followed by the 50th percentile of vegetation height (\emph{b} = -0.944, SE = 0.166, \emph{p} \textless{} 0.001, semi-partial \(R^2\) = 0.063). The percentage of explained variance did not decline with temporally misaligned LIDAR, nor was there a decrease in model performance as measured by AUC. AUC values were \textless{} 0.70 for all Swainson's Thrush models.

\hypertarget{winter-wren}{%
\subsection{Winter Wren}\label{winter-wren}}

Winter Wren occupancy was positively influenced by NDVI, the maximum vegetation height, and the percentage of first vegetation returns above the mean vegetation height. Winter Wren occupancy responded negatively to the canopy relief ratio and the density of vegetation returns between four and six meters in height. The model built using temporally aligned covariates explained 42\% of the variance in Winter Wren occupancy and had an AUC of 0.696. Maximum vegetation height contributed the most to model performance (\emph{b} = 1.024, SE = 0.27, \emph{p} \textless{} 0.001, semi-partial \(R^2\) = 0.047) followed by the canopy relief ratio (\emph{b} = -1.666, SE = 0.518, \emph{p} \textless{} 0.01, semi-partial \(R^2\) = 0.024). The percentage of explained variance did not decline with temporally misaligned LIDAR, nor was there a significant decrease in model performance as measured by AUC.

\hypertarget{white-throated-sparrow}{%
\subsection{White-Throated Sparrow}\label{white-throated-sparrow}}

The top White-Throated Sparrow model predicted that occupancy probability responds positively to NDVI, the coefficient of variation of vegetation height, and the maximum vegetation height, and negatively to the percentage of LiDAR vegetation returns above two meters. The model built using temporally aligned covariates explained 20\% of the variance in White-Throated Sparrow occupancy, and had an AUC of 0.705. Maximum vegetation height contributed the most to model performance (\emph{b} = 0.813, SE = 0.155, \emph{p} \textless{} 0.001, semi-partial \(R^2\) = 0.064) followed by NDVI (\emph{b} = 0.477, SE = 0.151, \emph{p} \textless{} 0.01, semi-partial \(R^2\) = 0.022). There was no discernible trend in the explained variance with increasing LiDAR temporal misalignment. However, we observed a significant decrease in model performance (\(R^2\)=0.28, \emph{p}\textless0.05). AUC statistics were \textless{} 0.7 with over five years of temporal misalignment.

\hypertarget{forest-age}{%
\subsection{Forest age}\label{forest-age}}

For American Redstart and Black-throated Green Warbler, we found a significant relationship between forest age and the pixel-wise differences between predictive maps (\emph{p}\textless001). Comparing the zero and fifteen year time lag models, we found that models using 15-year-old LiDAR data overestimated the occupancy probability of American Redstart in stands \textless25 years old. The 15-year-old LiDAR data overestimated the probability of Black-throated Green Warbler occupancy in forests of all ages (Figure \ref{fig:2scatter}). Forest age explained 23\% and 30\% of the variance between predictive maps for American Redstart and Black-throated Green Warbler, respectively.

\begin{figure}[!ht]
\includegraphics[width=1\linewidth,]{../3_output/figures/timelag_stats/spp_scatter_00to15} \caption{Scatter plots of per-pixel occupancy probability for predictive distribution maps representing zero and 15 years of time-lag between LIDAR and bird data. Scatter plots are coloured according to the forest age of each mapped pixel}\label{fig:2scatter}
\end{figure}

For Mourning Warbler, Swainson's Thrush, Winter Wren, and White-Throated Sparrow, we found significant but weak relationships between forest age and the differences between predictive maps built with increasing amounts of LiDAR-point count temporal misalignment (\emph{p}\textless001). For these species, forest age explained \textless{} 5\% of the variance between predictive maps.

\hypertarget{discussion}{%
\section{Discussion}\label{discussion}}

\hypertarget{model-performance}{%
\subsection{Model performance}\label{model-performance}}

We found LiDAR based models were moderately predictive of occupancy probability (0.7\textless AUC\textless-0.9) for four of the six focal species: American Redstart, Black-Throated Green Warbler, Mourning Warbler, and White-throated Sparrow. The influence of LiDAR time-lag with bird observations on SDM performance varied.

As predicted, SDMs for Mourning Warbler, an early-successional associate, saw significant declines in model performance with increased LiDAR temporal misalignment. Mourning Warbler nest and feed near the ground in dense shrub vegetation and colonize clearings opened by forestry and oil and gas exploration (\protect\hyperlink{ref-atwellSongbirdResponseExperimental2008}{Atwell et al., 2008}; \protect\hyperlink{ref-pitocchelliMourningWarblerGeothlypis2020}{Pitocchelli, 2020}). As forests regenerate, and canopy closing trees replace early-successional vegetation, Mourning Warbler abundance declines (as early as 10 years post-disturbance) (\protect\hyperlink{ref-brawnRoleDisturbanceEcology2001}{Brawn et al., 2001}). The proportion of LiDAR vegetation returns below two meters, an indicator of dense shrub understory vegetation, was the LiDAR variable that contributed the most to the explained variation in Mourning Warbler occupancy. This may help explain the declines in model performance with increased LiDAR temporal misalignment. As shrub density decreases through succession, LiDAR metrics indicating shrub becomes less useful. For Mourning Warbler, we found models became less predictive (AUC \textless{} -.70) with 13 years between LIDAR and bird surveys.

Temporal misalignment also strongly influenced the performance of SDMs for the White-throated Sparrow. The White-throated Sparrow is one of the most abundant species in Alberta's boreal mixed-wood forests (\protect\hyperlink{ref-Schmiegelow1997}{Schmiegelow et al., 1997}). They occur along forested edges, in early-successional stands or mature forest canopy gaps (\protect\hyperlink{ref-fallsWhitethroatedSparrowZonotrichia2020}{Falls and Kopachena, 2020}). White-throated Sparrows nest and feed near the ground with low dense vegetation cover (\protect\hyperlink{ref-fallsWhitethroatedSparrowZonotrichia2020}{Falls and Kopachena, 2020}). Similar to Mourning Warbler, it's feeding and nesting preferences likely impact the amount of acceptable time lag between LiDAR and bird survey data in predictive models because of changes in shrub layer vegetation occurring between the LiDAR acquisition and point-counts. The White-throated sparrow SDMs became less predictive (AUC \textless{} 0.70) after five years of time lag between LiDAR and point-count data.

We predicted that increasing the temporal lag between LiDAR and point-count data would lead to moderate declines in the performance of American Redstart SDMs. Our findings did not bear this out. Temporal misalignment had no discernible impact on model performance. American Redstart SDMs remained moderately predictive of occupancy with a 15 year gap between LIDAR and bird detections. American Redstart occurs in a range of successional stands and mixed-age plots (\protect\hyperlink{ref-sherryAmericanRedstartSetophaga2020a}{Sherry et al., 2020}). In Alberta, they are associated with structurally complex deciduous forests (\protect\hyperlink{ref-lestonLongtermChangesBoreal2018}{Leston et al., 2018}; \protect\hyperlink{ref-Mahon2016}{Mahon et al., 2016}). Our results support this. LiDAR measures of structural complexity were more predictive of American Redstart occupancy than other LiDAR variables. Two things may account for the American Restart models' resilience to LIDAR temporal misalignment. (1) The structurally complex, uneven-aged forests that American Redstart are associated with change over decades (\protect\hyperlink{ref-brassardStandStructureComposition2010}{Brassard and Chen, 2010}). The rate of change may not be captured with 15 years of time lag between LiDAR and point-counts. (2) Near Calling Lake, AB., the American Redstart decreases in abundance after harvesting (\protect\hyperlink{ref-nortonSongbirdResponsePartialcut1997}{Norton and Hannon, 1997}). Given that we controlled for natural and human disturbances occurring between LiDAR acquisition and bird surveys, American Redstart declines caused by harvesting were not captured by our models.

Contrary to our predictions, we observed a significant negative influence of LiDAR temporal misalignment on the performance of Black-throated Green Warbler SDMs. However, the SDMs were still moderately predictive of occupancy with fifteen years of time-lag between LIDAR and bird surveys. Canopy height was the biggest predictor of Black-throated Green Warbler occupancy. Black-throated Green Warbler is a forest interior species associated with older deciduous and mixed-wood forests (\protect\hyperlink{ref-morseBlackthroatedGreenWarbler2020a}{Morse and Poole, 2020}; Mahon2016, \protect\hyperlink{ref-Schieck2006}{Schieck and Song, 2006}). Declines in SDM performance were likely the result of changes in canopy height caused by natural gap opening events like individual tree fall or insect defoliation (\protect\hyperlink{ref-brassardStandStructureComposition2010}{Brassard and Chen, 2010}). However, canopy changes that occurred during our 15 year study period weren't large enough to reduce AUC to \textless0.70.

\hypertarget{recommendations}{%
\subsection{Recommendations}\label{recommendations}}

We identified two studies examining the influence of temporal misalignment between LiDAR and wildlife data on the performance of species-habitat models. Vierling et al. (\protect\hyperlink{ref-VierlingSwift2014}{2014}) studied the effect of six years of LiDAR time-lag with wildlife surveys on Brown Creeper (\emph{Certhia americana}) SDMs. They found that the six-year time-lag had only a small influence on model performance (a 5\% decrease in mapped occupancy probability). Similarly, Hill and Hinsley (\protect\hyperlink{ref-hillAirborneLidarWoodland2015}{2015}) examined how LiDAR data with a time-lag of up to 11 years with field data influenced breeding habitat models for the Great Tit (\emph{Parus major}). When comparing time-lags of one, four, and 11 years, they found only a small impact (less than 1\%) on model predictions. Both studies cautiously suggested that for mature and stable forests, temporal misalignment does not play a major role in the performance of predictive bird models. Brown Creeper occupies late-successional mature forests (\protect\hyperlink{ref-poulinBrownCreeperCerthia2020}{Poulin et al., 2020}) and the Great Tit is a habitat generalist (\protect\hyperlink{ref-vanbalenComparativeSudyBreeding1973}{Van Balen, 1973}). We found similar results for Black-throated Green Warbler and American Redstart. However, our results for early-successional species suggest that, for birds strongly associated with dense shrubs and open canopies, over five years of time-lag between LiDAR and wildlife surveys may erode model performance.

With its ability to capture vegetation structural attributes often missing from classified land-cover data, LiDAR is increasingly being used in bird studies (\protect\hyperlink{ref-Davies2014a}{Davies and Asner, 2014}; \protect\hyperlink{ref-Lefsky2002}{Lefsky et al., 2002}). Despite the increasing availability of LiDAR, multitemporal LiDAR data remains limited (\protect\hyperlink{ref-Lesak2011a}{Lesak et al., 2011}). Consequently, most studies modelling bird-habitat relationships include some temporal misalignment between LiDAR and bird data (\protect\hyperlink{ref-moudryRoleVegetationStructure2021}{Moudrý et al., 2021}): e.g., 3 years (\protect\hyperlink{ref-hinsleyApplicationLidarWoodland2006}{Hinsley et al., 2006}), 4 years (\protect\hyperlink{ref-Goetz2010}{Goetz et al., 2010}; \protect\hyperlink{ref-vogelerTerrainVegetationStructural2014a}{Vogeler et al., 2014}), 5 years (\protect\hyperlink{ref-Weisberg2014}{Weisberg et al., 2014}), and 10 years (\protect\hyperlink{ref-huberUsingRemotesensingData2016}{Huber et al., 2016}). Our results suggest researchers should consider temporal misalignment when applying LiDAR to bird-habitat models. For species associated primarily with mid- to late-successional boreal forests, coincident bird and LiDAR data may not be necessary. But caution should be taken with early-successional species occupying burned and harvested areas, and those that nest and feed near the ground with dense shrub vegetation. For these, we recommend limiting temporal misalignment to \textless5 years. If multi-temporal LiDAR is unavailable, other remote sensing may be better for characterizing post-disturbance vegetation, like time-series of spectral indices from optical satellites (\protect\hyperlink{ref-Kennedy2018}{Kennedy et al., 2018})

\hypertarget{conclusion}{%
\section{Conclusion}\label{conclusion}}

We evaluated how time lag between LiDAR acquisitions and bird surveys influenced model robustness for early-successional, mature forest, and forest generalist birds. We found that LIDAR-based SDMs are moderately predictive of occupancy for American Redstart, Black-throated Green Warbler, Mourning Warbler, and White-throated Sparrow. The influence of temporal misalignment on SDMs varied across species with the greatest impact on models for early-successional associates. For species occupying older, more stable forests, temporal misalignment between LiDAR and bird surveys had only a small impact on the predictive power of SDMs. For early-successional birds, our findings suggest that a time difference of 5-13 years between LIDAR and bird data may reduce model performance.

\pagebreak

\hypertarget{references}{%
\section*{References}\label{references}}
\addcontentsline{toc}{section}{References}

\hypertarget{refs}{}
\begin{CSLReferences}{1}{0}
\leavevmode\vadjust pre{\hypertarget{ref-abdalatiICESat2LaserAltimetry2010}{}}%
Abdalati, W., Zwally, H.J., Bindschadler, R., Csatho, B., Farrell, S.L., Fricker, H.A., Harding, D., Kwok, R., Lefsky, M., Markus, T., Marshak, A., Neumann, T., Palm, S., Schutz, B., Smith, B., Spinhirne, J., Webb, C., 2010. The {ICESat}-2 laser altimetry mission. Proceedings of the {IEEE} 98, 735--751. \url{https://doi.org/10.1109/JPROC.2009.2034765}

\leavevmode\vadjust pre{\hypertarget{ref-AESRD2013}{}}%
Alberta Environment and Sustainable Resource Development, 2013. General specifications for acquisition of {LiDAR} data. Government of Alberta, Edmonton, Alberta, Canada.

\leavevmode\vadjust pre{\hypertarget{ref-atwellSongbirdResponseExperimental2008}{}}%
Atwell, R.C., Schulte, L.A., Palik, B.J., 2008. Songbird response to experimental retention harvesting in red pine (pinus resinosa) forests. Forest Ecology and Management 255, 3621--3631.

\leavevmode\vadjust pre{\hypertarget{ref-babcockModelingForestBiomass2016}{}}%
Babcock, C., Finley, A.O., Cook, B.D., Weiskittel, A., Woodall, C.W., 2016. Modeling forest biomass and growth: Coupling long-term inventory and {LiDAR} data. Remote Sensing of Environment 182, 1--12. \url{https://doi.org/10.1016/j.rse.2016.04.014}

\leavevmode\vadjust pre{\hypertarget{ref-Bae2018}{}}%
Bae, S., Müller, J., Lee, D., Vierling, K.T., Vogeler, J.C., Vierling, L.A., Hudak, A.T., Latifi, H., Thorn, S., 2018. Taxonomic, functional, and phylogenetic diversity of bird assemblages are oppositely associated to productivity and heterogeneity in temperate forests. Remote Sensing of Environment 215, 145--156. \url{https://doi.org/10.1016/j.rse.2018.05.031}

\leavevmode\vadjust pre{\hypertarget{ref-Bae2014}{}}%
Bae, S., Reineking, B., Ewald, M., Mueller, J., 2014. Comparison of airborne lidar, aerial photography, and field surveys to model the habitat suitability of a cryptic forest species -- the hazel grouse. International Journal of Remote Sensing 35, 6469--6489. \url{https://doi.org/10.1080/01431161.2014.955145}

\leavevmode\vadjust pre{\hypertarget{ref-bartonMuMInMultimodelInference2020}{}}%
Bartoń, K., 2020. \href{https://CRAN.R-project.org/package=MuMIn}{{MuMIn}: Multi-model inference} (manual).

\leavevmode\vadjust pre{\hypertarget{ref-batesFittingLinearMixedeffects2015}{}}%
Bates, D., Mächler, M., Bolker, B., Walker, S., 2015. Fitting linear mixed-effects models using {\textless{}}span class="nocase"{\textgreater{}}lme4{\textless{}}/span{\textgreater{}}. Journal of Statistical Software 67, 1--48. \url{https://doi.org/10.18637/jss.v067.i01}

\leavevmode\vadjust pre{\hypertarget{ref-Bergeron2012}{}}%
Bergeron, Y., Fenton, N.J., 2012. Boreal forests of eastern canada revisited: Old growth, nonfire disturbances, forest succession, and biodiversity. Botany 90, 509--523. \url{https://doi.org/10.1139/B2012-034}

\leavevmode\vadjust pre{\hypertarget{ref-BAM2018}{}}%
Boreal Avian Modelling Project (BAM), 2018. Boreal avian modelling project {[}WWW Document{]}. URL \url{http://www.borealbirds.ca/} (accessed 2.22.2018).

\leavevmode\vadjust pre{\hypertarget{ref-Bradbury2005}{}}%
Bradbury, R.B., Hill, R.A., Mason, D.C., Hinsley, S.A., Wilson, J.D., Balzter, H., Anderson, G.Q.A., Whittingham, M.J., Davenport, I.J., Bellamy, P.E., 2005. Modelling relationships between birds and vegetation structure using airborne {LiDAR} data: A review with case studies from agricultural and woodland environments. Ibis 147, 443--452. \url{https://doi.org/10.1111/j.1474-919x.2005.00438.x}

\leavevmode\vadjust pre{\hypertarget{ref-Brandt2013}{}}%
Brandt, J.P., Flannigan, M.D., Maynard, D.G., Thompson, I.D., Volney, W.J.A., 2013. An introduction to canada's boreal zone: Ecosystem processes, health, sustainability, and environmental issues. Environmental Reviews 21, 207--226. \url{https://doi.org/10.1139/er-2013-0040}

\leavevmode\vadjust pre{\hypertarget{ref-brassardStandStructureComposition2010}{}}%
Brassard, B.W., Chen, H.Y.H., 2010. Stand structure and composition dynamics of boreal mixedwood forest: Implications for forest management.

\leavevmode\vadjust pre{\hypertarget{ref-brawnRoleDisturbanceEcology2001}{}}%
Brawn, J.D., Robinson, S.K., Thompson III, F.R., 2001. The role of disturbance in the ecology and conservation of birds. Annual review of Ecology and Systematics 32, 251--276.

\leavevmode\vadjust pre{\hypertarget{ref-burnhamModelSelectionMultimodel2002}{}}%
Burnham, K.P., Anderson, D.R., 2002. Model selection and multimodel inference: A practical information-theoretic approach. Springer New York.

\leavevmode\vadjust pre{\hypertarget{ref-CarrascalDiaz2006}{}}%
Carrascal, L., Diaz, L., 2006. \href{https://www.tandfonline.com/doi/abs/10.2980/1195-6860(2006)13\%5B100\%3AWBDIAA\%5D2.0.CO\%3B2}{Winter bird distribution in abiotic and habitat structural gradients: A case study with mediterranean montane oakwoods}. Ecoscience 13, 100--110.

\leavevmode\vadjust pre{\hypertarget{ref-Carrillo-Rubio2014}{}}%
Carrillo-Rubio, E., Kéry, M., Morreale, S.J., Sullivan, P.J., Gardner, B., Cooch, E.G., Lassoie, J.P., 2014. Use of multispecies occupancy models to evaluate the response of bird communities to forest degradation associated with logging. Conservation Biology 28, 1034--1044. \url{https://doi.org/10.1111/cobi.12261}

\leavevmode\vadjust pre{\hypertarget{ref-clawgesUseAirborneLidar2008}{}}%
Clawges, R., Vierling, K., Vierling, L., Rowell, E., 2008. The use of airborne lidar to assess avian species diversity, density, and occurrence in a pine/aspen forest. Remote Sensing of Environment 112, 2064--2073. \url{https://doi.org/10.1016/j.rse.2007.08.023}

\leavevmode\vadjust pre{\hypertarget{ref-coopsForestStructureHabitat2016}{}}%
Coops, N.C., Tompaski, P., Nijland, W., Rickbeil, G.J.M., Nielsen, S.E., Bater, C.W., Stadt, J.J., 2016. A forest structure habitat index based on airborne laser scanning data. Ecological Indicators 67, 346--357. \url{https://doi.org/10.1016/j.ecolind.2016.02.057}

\leavevmode\vadjust pre{\hypertarget{ref-Cumming2011a}{}}%
Cosco, J.A., 2011. Common attribute schema ({CAS}) for forest inventories across canada. Boreal Avian Modelling Project; Canadian {BEACONs} Project, Edmonton, Alberta, Canada.

\leavevmode\vadjust pre{\hypertarget{ref-Davies2014a}{}}%
Davies, A.B., Asner, G.P., 2014. Advances in animal ecology from 3D-{LiDAR} ecosystem mapping. Trends in Ecology and Evolution 29, 681--691. \url{https://doi.org/10.1016/j.tree.2014.10.005}

\leavevmode\vadjust pre{\hypertarget{ref-dubayahGlobalEcosystemDynamics2020a}{}}%
Dubayah, R., Blair, J.B., Goetz, S., Fatoyinbo, L., Hansen, M., Healey, S., Hofton, M., Hurtt, G., Kellner, J., Luthcke, S., Armston, J., Tang, H., Duncanson, L., Hancock, S., Jantz, P., Marselis, S., Patterson, P.L., Qi, W., Silva, C., 2020. The global ecosystem dynamics investigation: High-resolution laser ranging of the earth's forests and topography. Science of Remote Sensing 1, 100002. https://doi.org/\url{https://doi.org/10.1016/j.srs.2020.100002}

\leavevmode\vadjust pre{\hypertarget{ref-englerAvianSDMsCurrent2017}{}}%
Engler, J.O., Stiels, D., Schidelko, K., Strubbe, D., Quillfeldt, P., Brambilla, M., 2017. Avian {SDMs}: Current state, challenges, and opportunities. Journal of Avian Biology 48, 1483--1504. \url{https://doi.org/10.1111/jav.01248}

\leavevmode\vadjust pre{\hypertarget{ref-fallsWhitethroatedSparrowZonotrichia2020}{}}%
Falls, J., Kopachena, J., 2020. White-throated sparrow (zonotrichia albicollis), version 1.0, in: Birds of the World (a. F. Poole, Editor). Cornell Lab of Ornithology, Ithaca, {NY}, {USA}.

\leavevmode\vadjust pre{\hypertarget{ref-farrellUsingLiDARderivedVegetation2013b}{}}%
Farrell, S.L., Collier, B.A., Skow, K.L., Long, A.M., Campomizzi, A.J., Morrison, M.L., Hays, K.B., Wilkins, R.N., 2013. Using {LiDAR}-derived vegetation metrics for high-resolution, species distribution models for conservation planning. Ecosphere 4, 42. \url{https://doi.org/10.1890/ES12-000352.1}

\leavevmode\vadjust pre{\hypertarget{ref-ficetolaHowManyPredictors2014}{}}%
Ficetola, G.F., Bonardi, A., Mücher, C.A., Gilissen, N.L.M., Padoa-Schioppa, E., 2014. How many predictors in species distribution models at the landscape scale? Land use versus {LiDAR}-derived canopy height. International Journal of Geographical Information Science 28, 1723--1739. \url{https://doi.org/10.1080/13658816.2014.891222}

\leavevmode\vadjust pre{\hypertarget{ref-fogaCloudDetectionAlgorithm2017}{}}%
Foga, S., Scaramuzza, P.L., Guo, S., Zhu, Z., Dilley, R.D., Jr., Beckmann, T., Schmidt, G.L., Dwyer, J.L., Hughes, M.J., Laue, B., 2017. Cloud detection algorithm comparison and validation for operational landsat data products. Remote Sensing of Environment 194, 379--390. \url{https://doi.org/10.1016/j.rse.2017.03.026}

\leavevmode\vadjust pre{\hypertarget{ref-fourcadePaintingsPredictDistribution2018}{}}%
Fourcade, Y., Besnard, A.G., Secondi, J., 2018. Paintings predict the distribution of species, or the challenge of selecting environmental predictors and evaluation statistics. Global Ecology and Biogeography 27, 245--256. \url{https://doi.org/10.1111/geb.12684}

\leavevmode\vadjust pre{\hypertarget{ref-franklinMappingSpeciesDistributions2010}{}}%
Franklin, J., 2010. Mapping species distributions: Spatial inference and prediction. Cambridge University Press.

\leavevmode\vadjust pre{\hypertarget{ref-Franklin1995}{}}%
Franklin, J., 1995. Predictive vegetation mapping: Geographic modelling of biospatial patterns in relation to environmental gradients. Progress in Physical Geography: Earth and Environment 19, 474--499. \url{https://doi.org/10.1177/030913339501900403}

\leavevmode\vadjust pre{\hypertarget{ref-gauthierBorealForestHealth2015}{}}%
Gauthier, S., Bernier, P., Kuuluvainen, T., Shvidenko, A.Z., Schepaschenko, D.G., 2015. Boreal forest health and global change. {SCIENCE} 349, 819--822. \url{https://doi.org/10.1126/science.aaa9092}

\leavevmode\vadjust pre{\hypertarget{ref-Goetz2010}{}}%
Goetz, S.J., Steinberg, D., Betts, M.G., Holmes, R.T., Doran, P.J., Dubayah, R., Hofton, M., 2010. Lidar remote sensing variables predict breeding habitat of a neotropical migrant bird. Ecology 91, 1569--1576. \url{https://doi.org/10.1890/09-1670.1}

\leavevmode\vadjust pre{\hypertarget{ref-gorelickGoogleEarthEngine2017}{}}%
Gorelick, N., Hancher, M., Dixon, M., Ilyushchenko, S., Thau, D., Moore, R., 2017. Google earth engine: Planetary-scale geospatial analysis for everyone. Remote sensing of Environment 202, 18--27. \url{https://doi.org/10.1016/j.rse.2017.06.031}

\leavevmode\vadjust pre{\hypertarget{ref-GotmarkBlomqvist1995}{}}%
Gotmark, F., Blomqvist, D., Johansson, O.C., Bergkvist, J., 1995. Nest site selection: A trade-off between concealment and view of the surroundings? Journal of Avian Biology 26, 305. \url{https://doi.org/10.2307/3677045}

\leavevmode\vadjust pre{\hypertarget{ref-Guisan2005}{}}%
Guisan, A., Thuiller, W., 2005. Predicting species distribution: Offering more than simple habitat models. Ecology Letters 8, 993--1009. \url{https://doi.org/10.1111/j.1461-0248.2005.00792.x}

\leavevmode\vadjust pre{\hypertarget{ref-guisanPredictiveHabitatDistribution2000}{}}%
Guisan, A., Zimmermann, N.E., 2000. Predictive habitat distribution models in ecology. Ecological Modelling 135, 147--186. \url{https://doi.org/10.1016/S0304-3800(00)00354-9}

\leavevmode\vadjust pre{\hypertarget{ref-halajImportanceHabitatStructure2000}{}}%
Halaj, J., Ross, D., Moldenke, A., 2000. Importance of habitat structure to the arthropod food-web in douglas-fir canopies. {OIKOS} 90, 139--152. \url{https://doi.org/10.1034/j.1600-0706.2000.900114.x}

\leavevmode\vadjust pre{\hypertarget{ref-hannonCorridorsMayNot2002}{}}%
Hannon, S., Schmiegelow, F., 2002. Corridors may not improve the conservation value of small reserves for most boreal birds. Ecological Applications 12, 1457--1468.

\leavevmode\vadjust pre{\hypertarget{ref-He2015}{}}%
He, K.S., Bradley, B.A., Cord, A.F., Rocchini, D., Tuanmu, M.-N.N., Schmidtlein, S., Turner, W., Wegmann, M., Pettorelli, N., 2015. Will remote sensing shape the next generation of species distribution models? Remote Sensing in Ecology and Conservation 1, 4--18. \url{https://doi.org/10.1002/rse2.7}

\leavevmode\vadjust pre{\hypertarget{ref-hijmansRasterGeographicData2021}{}}%
Hijmans, R.J., 2021. \href{https://CRAN.R-project.org/package=raster}{Raster: Geographic data analysis and modeling} (manual).

\leavevmode\vadjust pre{\hypertarget{ref-R-raster}{}}%
Hijmans, R.J., 2020. \href{https://CRAN.R-project.org/package=raster}{Raster: Geographic data analysis and modeling} (manual).

\leavevmode\vadjust pre{\hypertarget{ref-hillAirborneLidarWoodland2015}{}}%
Hill, R.A., Hinsley, S.A., 2015. Airborne lidar for woodland habitat quality monitoring: Exploring the significance of lidar data characteristics when modelling organism-habitat relationships. Remote Sensing 7, 3446--3466. \url{https://doi.org/10.3390/rs70403446}

\leavevmode\vadjust pre{\hypertarget{ref-hinsleyApplicationLidarWoodland2006}{}}%
Hinsley, S.A., Hill, R.A., Bellamy, P.E., Balzter, H., 2006. The application of lidar in woodland bird ecology: Climate, canopy structure, and habitat quality. {PHOTOGRAMMETRIC} {ENGINEERING} {AND} {REMOTE} {SENSING} 72, 1399--1406. \url{https://doi.org/10.14358/PERS.72.12.1399}

\leavevmode\vadjust pre{\hypertarget{ref-huberUsingRemotesensingData2016}{}}%
Huber, N., Kienast, F., Ginzler, C., Pasinelli, G., 2016. Using remote-sensing data to assess habitat selection of a declining passerine at two spatial scales. Landscape Ecology 31, 1919--1937. \url{https://doi.org/10.1007/s10980-016-0370-1}

\leavevmode\vadjust pre{\hypertarget{ref-jaegerR2glmmComputesSquared2017}{}}%
Jaeger, B., 2017. \href{https://CRAN.R-project.org/package=r2glmm}{r2glmm: Computes r squared for mixed (multilevel) models} (manual).

\leavevmode\vadjust pre{\hypertarget{ref-Kennedy2018}{}}%
Kennedy, R.E., Yang, Z., Gorelick, N., Braaten, J., Cavalcante, L., Cohen, W.B., Healey, S., 2018. Implementation of the {LandTrendr} algorithm on google earth engine. Remote Sensing 10, 691. \url{https://doi.org/10.3390/rs10050691}

\leavevmode\vadjust pre{\hypertarget{ref-Kortmann2018}{}}%
Kortmann, M., Heurich, M., Latifi, H., Roesner, S., Seidl, R., Mueller, J., Thorn, S., Rösner, S., Seidl, R., Müller, J., Thorn, S., 2018. Forest structure following natural disturbances and early succession provides habitat for two avian flagship species, capercaillie (tetrao urogallus) and hazel grouse (tetrastes bonasia). Biological Conservation 226, 81--91. \url{https://doi.org/10.1016/j.biocon.2018.07.014}

\leavevmode\vadjust pre{\hypertarget{ref-Lefsky2002}{}}%
Lefsky, M.A., Cohen, W.B., Parker, G.G., Harding, D.J., Parker, G.G., Harding, D.J., 2002. Lidar remote sensing for ecosystem studies. {BioScience} 52, 19--30. \url{https://doi.org/10.1641/0006-3568(2002)052\%5B0019:lrsfes\%5D2.0.co;2}

\leavevmode\vadjust pre{\hypertarget{ref-Lesak2011a}{}}%
Lesak, A.A., Radeloff, V.C., Hawbaker, T.J., Pidgeon, A.M., Gobakken, T., Contrucci, K., 2011. Modeling forest songbird species richness using {LiDAR}-derived measures of forest structure. Remote Sensing of Environment 115, 2823--2835. \url{https://doi.org/10.1016/j.rse.2011.01.025}

\leavevmode\vadjust pre{\hypertarget{ref-lestonLongtermChangesBoreal2018}{}}%
Leston, L., Bayne, E., Schmiegelow, F., 2018. Long-term changes in boreal forest occupancy within regenerating harvest units. Forest Ecology and Management 421, 40--53. \url{https://doi.org/10.1016/J.FORECO.2018.02.029}

\leavevmode\vadjust pre{\hypertarget{ref-limLiDARRemoteSensing2003}{}}%
Lim, K., Treitz, P., Wulder, M., St-Onge, B., Flood, M., 2003. {LiDAR} remote sensing of forest structure. Progress in Physical Geography 27, 88--106. \url{https://doi.org/10.1191/0309133303pp360ra}

\leavevmode\vadjust pre{\hypertarget{ref-MacArthurMacArthur1961}{}}%
MacArthur, R.H., MacArthur, J.W., 1961. On bird species diversity. Ecology 42, 594--598. \url{https://doi.org/10.2307/1932254}

\leavevmode\vadjust pre{\hypertarget{ref-Mahon2016}{}}%
Mahon, C.L., Holloway, G., Sólymos, P., Cumming, S.G., Bayne, E.M., Schmiegelow, F.K.A., Song, S.J., 2016. Community structure and niche characteristics of upland and lowland western boreal birds at multiple spatial scales. Forest Ecology and Management 361, 99--116. \url{https://doi.org/10.1016/j.foreco.2015.11.007}

\leavevmode\vadjust pre{\hypertarget{ref-mccarthy2001gap}{}}%
McCarthy, J., 2001. Gap dynamics of forest trees: A review with particular attention to boreal forests. Environmental reviews 9, 1--59.

\leavevmode\vadjust pre{\hypertarget{ref-mcgaugheyFUSIONLDVSoftware2018}{}}%
McGaughey, R.J., 2018. {FUSION}/{LDV}: Software for {LIDAR} data analysis and visualization. United States Department of Agriculture, Pacific Northwest Research Station, Seattle, {USA}.

\leavevmode\vadjust pre{\hypertarget{ref-morseBlackthroatedGreenWarbler2020a}{}}%
Morse, D., Poole, A., 2020. Black-throated green warbler (setophaga virens), version 1.0, in: Birds of the World (p. G. Rodewald, Editor). Cornell Lab of Ornithology, Ithaca, {NY}, {USA}.

\leavevmode\vadjust pre{\hypertarget{ref-moudryRoleVegetationStructure2021}{}}%
Moudrý, V., Moudrá, L., Barták, V., Bejček, V., Gdulová, K., Hendrychová, M., Moravec, D., Musil, P., Rocchini, D., Šťastný, K., Volf, O., Šálek, M., 2021. The role of the vegetation structure, primary productivity and senescence derived from airborne {LiDAR} and hyperspectral data for birds diversity and rarity on a restored site. Landscape and Urban Planning 210, 104064. \url{https://doi.org/10.1016/j.landurbplan.2021.104064}

\leavevmode\vadjust pre{\hypertarget{ref-nakagawaGeneralSimpleMethod2013}{}}%
Nakagawa, S., Schielzeth, H., 2013. A general and simple method for obtaining R2 from generalized linear mixed-effects models. Methods in Ecology and Evolution 4, 133--142. \url{https://doi.org/10.1111/j.2041-210x.2012.00261.x}

\leavevmode\vadjust pre{\hypertarget{ref-nortonSongbirdResponsePartialcut1997}{}}%
Norton, M.R., Hannon, S.J., 1997. Songbird response to partial-cut logging in the boreal mixedwood forest of alberta. Canadian Journal of Forest Research 27, 44--53. \url{https://doi.org/10.1139/x96-149}

\leavevmode\vadjust pre{\hypertarget{ref-pettorelliNormalizedDifferenceVegetation2011}{}}%
Pettorelli, N., Ryan, S., Mueller, T., Bunnefeld, N., Jedrzejewska, B., Lima, M., Kausrud, K., 2011. The normalized difference vegetation index ({NDVI}): Unforeseen successes in animal ecology. Climate Research 46, 15--27. \url{https://doi.org/10.3354/cr00936}

\leavevmode\vadjust pre{\hypertarget{ref-pitocchelliMourningWarblerGeothlypis2020}{}}%
Pitocchelli, J., 2020. Mourning warbler (geothlypis philadelphia), version 1.0, in: Birds of the World (p. G. Rodewald, Editor). Cornell Lab of Ornithology, Ithaca, {NY}, {USA}.

\leavevmode\vadjust pre{\hypertarget{ref-poulinBrownCreeperCerthia2020}{}}%
Poulin, J., D'Astous, E., Villard, M., Hejl, S.J., Newlon, K.R., McFadzen, M.E., Young, J.S., Ghalambor, C.K., 2020. Brown creeper (certhia americana), version 1.0, in: Poole, A. (Ed.), Birds of the World. Cornell Lab of Ornithology, Ithaca, {NY}, {USA}.

\leavevmode\vadjust pre{\hypertarget{ref-R-base}{}}%
R Core Team, 2020. \href{https://www.R-project.org/}{R: A language and environment for statistical computing} (manual). Vienna, Austria.

\leavevmode\vadjust pre{\hypertarget{ref-Renner2018}{}}%
Renner, S.C., Suarez-Rubio, M., Kaiser, S., Nieschulze, J., Kalko, E.K.V.V., Tschapka, M., Jung, K., 2018. Divergent response to forest structure of two mobile vertebrate groups. Forest Ecology and Management 415-416, 129--138. \url{https://doi.org/10.1016/j.foreco.2018.02.028}

\leavevmode\vadjust pre{\hypertarget{ref-robinPROCOpensourcePackage2011}{}}%
Robin, X., Turck, N., Hainard, A., Tiberti, N., Lisacek, F., Sanchez, J.-C., Müller, M., 2011. {pROC}: An open-source package for r and s+ to analyze and compare {ROC} curves. {BMC} Bioinformatics 12, 77.

\leavevmode\vadjust pre{\hypertarget{ref-Schieck2006}{}}%
Schieck, J., Song, S.J., 2006. Changes in bird communities throughout succession following fire and harvest in boreal forests of western north america: Literature review and meta-analyses. Canadian Journal of Forest Research 36, 1299--1318. \url{https://doi.org/10.1139/X06-017}

\leavevmode\vadjust pre{\hypertarget{ref-Schmiegelow1997}{}}%
Schmiegelow, F.K., Machtans, C., Hannon, S., 1997. Are boreal birds resilient to forest fragmentation? An experimental study of short-term community responses. Ecology 78, 1914--1932. \url{https://doi.org/10.1890/0012-9658(1997)078\%5B1914:abbrtf\%5D2.0.co;2}

\leavevmode\vadjust pre{\hypertarget{ref-sherryAmericanRedstartSetophaga2020a}{}}%
Sherry, T., Holmes, R., Pyle, P., Patten, M., Rodewald, P., 2020. American redstart (setophaga ruticilla), version 1.0, in: Birds of the World (p. G. Rodewald, Editor). Cornell Lab of Ornithology, Ithaca, {NY}, {USA}.

\leavevmode\vadjust pre{\hypertarget{ref-sittersAssociationsOccupancyHabitat2014}{}}%
Sitters, H., Christie, F., Di Stefano, J., Swan, M., Collins, P., York, A., 2014. Associations between occupancy and habitat structure can predict avian responses to disturbance: Implications for conservation management. {FOREST} {ECOLOGY} {AND} {MANAGEMENT} 331, 227--236. \url{https://doi.org/10.1016/j.foreco.2014.08.013}

\leavevmode\vadjust pre{\hypertarget{ref-SolymosMatsuoka2013}{}}%
Sólymos, P., Matsuoka, S.M., Bayne, E.M., Lele, S.R., Fontaine, P., Cumming, S.G., Stralberg, D., Schmiegelow, F.K.A., Song, S.J., 2013. Calibrating indices of avian density from non-standardized survey data: Making the most of a messy situation. Methods in Ecology and Evolution 4, 1047--1058. \url{https://doi.org/10.1111/2041-210X.12106}

\leavevmode\vadjust pre{\hypertarget{ref-geologicalsurveyLandsat47Surface2018}{}}%
Survey, U.S.G., 2018. Landsat 4-7 surface reflectance (ledaps) product guide. Department of the Interior, U.S. Geological Survey, Sioux Falls, {SD}, {USA}.

\leavevmode\vadjust pre{\hypertarget{ref-USGS_NDVI}{}}%
USGS, n.d. Landsat normalized difference vegetation index {[}WWW Document{]}. URL \url{https://www.usgs.gov/land-resources/nli/landsat/landsat-normalized-difference-vegetation-index?qt-science_support_page_related_con=0\#qt-science_support_page_related_con} (accessed 4.28.2020).

\leavevmode\vadjust pre{\hypertarget{ref-valbuenaStandardizingEcosystemMorphological2020}{}}%
Valbuena, R., O'Connor, B., Zellweger, F., Simonson, W., Vihervaara, P., Maltamo, M., Silva, C.A., Almeida, D.R., Danks, F., Morsdorf, F., 2020. Standardizing ecosystem morphological traits from 3D information sources. Trends in Ecology \& Evolution 35, 656--667. \url{https://doi.org/10.1016/j.tree.2020.03.006}

\leavevmode\vadjust pre{\hypertarget{ref-vanbalenComparativeSudyBreeding1973}{}}%
Van Balen, J.H., 1973. A comparative sudy of the breeding ecology of the great tit parus major in different habitats. Ardea 55, 1--93.

\leavevmode\vadjust pre{\hypertarget{ref-vanagasReceiverOperatingCharacteristic2004}{}}%
Vanagas, G., 2004. Receiver operating characteristic curves and comparison of cardiac surgery risk stratification systems. Interactive {CardioVascular} and Thoracic Surgery 3, 319--322. \url{https://doi.org/10.1016/j.icvts.2004.01.008}

\leavevmode\vadjust pre{\hypertarget{ref-Vaughn2003}{}}%
Vaughn, I.P., Ormerod, S.J., 2003. Improving the quality of distribution models for conservation by addressing shortcomings in the field collection of training data. Conservation Biology 17, 1601--1611. \url{https://doi.org/10.1111/j.1523-1739.2003.00359.x}

\leavevmode\vadjust pre{\hypertarget{ref-VierlingSwift2014}{}}%
Vierling, K.T., Swift, C.E., Hudak, A.T., Vogeler, J.C., Vierling, L.A., 2014. How much does the time lag between wildlife field-data collection and {LiDAR}-data acquisition matter for studies of animal distributions? A case study using bird communities. Remote Sensing Letters 5, 185--193. \url{https://doi.org/10.1080/2150704X.2014.891773}

\leavevmode\vadjust pre{\hypertarget{ref-vogelerTerrainVegetationStructural2014a}{}}%
Vogeler, J.C., Hudak, A.T., Vierling, L.A., Evans, J., Green, P., Vierling, K.I.T., 2014. Terrain and vegetation structural influences on local avian species richness in two mixed-conifer forests. Remote Sensing of Environment 147, 13--22. \url{https://doi.org/10.1016/j.rse.2014.02.006}

\leavevmode\vadjust pre{\hypertarget{ref-Weisberg2014}{}}%
Weisberg, P.J., Dilts, T.E., Becker, M.E., Young, J.S., Wong-Kone, D.C., Newton, W.E., Ammon, E.M., 2014. Guild-specific responses of avian species richness to {LiDAR}-derived habitat heterogeneity. Acta Oecologica 59, 72--83. \url{https://doi.org/10.1016/j.actao.2014.06.002}

\end{CSLReferences}

\pagebreak

\end{document}
